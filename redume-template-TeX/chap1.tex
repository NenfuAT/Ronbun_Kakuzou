\section{はじめに}
近年ではスマートウォッチと
呼ばれる腕時計型のデバイスやセンサグローブと呼ばれる手袋型のデバイスなど動作を取得できるウェアラブルデバイスが普及し始めている.
それに伴い装着者の腕の動きを容易に取得可能となった.
ウェアラブルセンサを用いたヤスリがけ動作の技能評価による熟達者の動作の再現に関する研究
\cite{ウェアラブルセンサを用いた熟練指導員のヤスリがけ技能主観評価値の再現}
のように専門的な動作の分析が盛んに行われている.

専門的な動作の技能評価は進んできているが,日常生活に関係してくる動作を評価しているものはまだ少ない.
日常動作の中で技能が必要となってくる動作の一つに調理があげられる.
例として,包丁に直接加速度センサを取り付け包丁技術を判定する研究\cite{加速度センサを用いた包丁技術向上支援システムの提案}がある.
調理動作分析の研究は主に調理機材やキッチン,人間にセンサ類を取り付けるなど,自作の装置を用いて調理動作を推定している場合が多い.
また,ウェアラブルデバイスを用いた調理の切る動作の分析の研究がある\cite{kumazawaanalysis}.
しかし,この研究では分析の際に加速度しか使用していない.
そこで本研究では包丁を持った手の3次元的な動きをセンシングした加速度・角速度を用いて算出し,特徴量の抽出を目的とする.
抽出した特徴量は過去の自分との比較や熟練者との比較により動作の改善を促進し,使用者の調理技能向上に使用できる.
