\section{今後の課題}

今後の課題は以下の3つである.
一つ目については,今回の調査では特徴量を抽出したのみで他者との比較から調理技能の向上につながるかは検証できていない.
% 検証のためには他者との比較から調理に対する意識の変化など調理技能の向上と関係する変化が見られるか実験を行う必要がある.
二つ目の他の調理行動への応用については特徴量の抽出は切る動作しか行っていないため他の調理動作でも特徴量を算出できるかについて調査する必要がある.
% 炒める工程・混ぜる工程などがあるためそれぞれについて調査する.
最後に,包丁を持つ際などの動きが切っていると誤判定される問題がある.
% そのため,センシング開始前に包丁をあらかじめ持っておくなどの対策を行う必要がある.
% その際安全性にも考慮する必要があるため切り始める前と後に包丁を持った状態でも行える安全なジェスチャーを考える必要がある.

% また今後の展望として特徴量を抽出する際に使用した3次元空間での可視化システムは結果のフィードバックへの応用を検討する.
% 抽出した特徴量のみでなく調理動作そのものを可視化して他者との比較を行う.
% これにより特徴量を可視化するだけの場合と意識の変化に違いが見られるか検討できる.