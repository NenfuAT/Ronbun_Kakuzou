%
%  愛知工業大学 情報科学部 情報科学科
%    要旨集用LaTeXテンプレート(2016.11.28)
%
%  original by Nobuhiro Ito
%  revised by Susumu Suzuki & Masashi Morimoto on 2016.11.28
%
\documentclass{jarticle}
\pagestyle{empty}

%% レイアウト
\setlength{\topmargin}{-10.4mm}
\setlength{\headheight}{0mm}
\setlength{\headsep}{0mm}
\setlength{\textheight}{262mm}
\setlength{\textwidth}{180mm}
%\setlength{\topskip}{7mm}
\setlength{\evensidemargin}{-10.4mm} 
\setlength{\oddsidemargin}{-10.4mm} 
\setlength{\columnsep}{8mm}

%% パッケージ環境
% graphicx: 画像読み込み
%  dvipdfmxをドライバ指定することでpdf/png/jpeg形式の図を利用可能
%  異なるドライバを利用する場合はそのドライバ指定に変更
% \usepackage{graphicx} % suzuki
\usepackage[dvipdfmx]{graphicx} % suzuki
% subcaption: subfigureを置き換えたパッケージ(複数の図用)
% \setlength{\footskip}{12mm}
% \usepackage{subfigure}	 % suzuki
\usepackage{subcaption} % suzuki
% color:文章に色をつけたいとき
%  利用例:\textcolor{red}{文章}
% \usepackage{color}
\usepackage{booktabs}
\usepackage{setspace}

% \figimage コマンドの定義
%{画像のパス}{横幅(mm)}{キャプション}{ラベル名}
\newcommand{\figimage}[4]{
	\begin{figure}[tbh]
		\centering
		\includegraphics[width=#2 mm]{#1}
		\caption{#3}
		\label{#4}
	\end{figure}
}
% ---------------


% 行間調整
\setstretch{0.9}

%sectionのフォントサイズ修正
\makeatletter
\def\section{\@startsection {section}{1}{\z@}{2.5ex plus -1ex minus -.2ex}{1.3 ex plus .1ex}{\large\bf}}
\makeatother 

%subsectionのフォントサイズ修正
\makeatletter
\def\subsection{\@startsection {subsection}{1}{\z@}{1.5ex plus -1ex minus -.4ex}{0.3 ex plus .1ex}{\bf}}
\makeatother 

\begin{document}
\twocolumn[

\begin{center}
%タイトル
{\LARGE \textbf{手首の3次元的な動きの分析に基づく\\刺身を切る動作の特徴量抽出に関する研究}}\\
%サブタイトル
%{\Large \textbf{必要に応じてサブタイトル}}
\end{center}

\begin{center}
% 著者
\begin{tabular}{cccc}
% 1名の場合
%\multicolumn{4}{c}{K11001 愛工総和}\\
% 2名の場合
%& K11002 愛工七音 & X11003 愛工頼音 &\\
% 3名の場合
%K11001 愛工総和 & K11002 愛工今鹿 & X11003 愛工姫星&\\
% 4名の場合
%K11001 愛工総和 & K11002 愛工今鹿 & X11003 愛工姫星& X11004 愛工緑輝\\
\multicolumn{4}{c}{K21091 西幸希}\\
% 指導教員
\multicolumn{4}{c}{\textbf{指導教員} 梶克彦}
\end{tabular}
\hspace{2zw}
\end{center}
]

%--------------------------------------------
\chapter{はじめに}
\section{背景}
近年ではスマートウォッチと呼ばれる腕時計型のデバイスやセンサグローブと呼ばれる手袋型のデバイスなど,動作を取得できるウェアラブルデバイスが普及し始めている.
これらのデバイスは,小型化と高性能化が進み,一般消費者にも手軽に利用可能となりつつあり,装着者の腕や手の動きを容易かつ詳細に取得できる.
総務省の調査3ではスマートフォンの世帯保有率は年々増加しており,2022 年では 90.1\%となっている.
MMD研究所のスマートウォッチの所持率調査では2019年4は18.0\%(n=1867)に対して2021年5は38.0\%(n=658)とスマートウォッチも普及してきているのがわかる.
またスマートウォッチについては内蔵されているセンサの数も増加してきている.
現行のスマートウォッチであるPixelWatch2の場合GPS・コンパス・高度計・酸素飽和度計測用赤色および赤外線センサ・多目的電気センサ・マルチパス光学式心拍数センサ・3軸加速度計・ジャイロスコープ・周囲光センサ・皮膚コンダクタンスを測定する電気センサ・皮膚温センサ・気圧計・磁力計が内蔵されておりスマートフォンに引けを取らないセンサの量がある.

取得されたデータは,単なる動作記録にとどまらず,さまざまな用途で活用されている.例えば,スポーツ分野ではアスリートの動きを詳細に計測し,そのデータをプロの動作と比較してフォーム改善によるパフォーマンス向上や,怪我のリスク低減を目指している.
さらに,医療分野では,患者のリハビリテーション支援や日常生活動作の監視にウェアラブルデバイスが利用されており,歩行解析や関節可動域の測定を通じて回復状況を定量的に評価している.
これにより,従来の主観的な評価に比べて,より客観的で精密な技能評価が実現しており,リハビリプランの最適化や治療効果の測定に役立てられている.

このように,スポーツや医療分野では専門的な動作の技能評価が進んでいる一方で,日常生活に関連する動作の評価については,まだ十分な研究が進んでいない.
日常動作の中でも技能が必要とされるものとして,調理動作が挙げられる.調理は単なる日常作業ではなく,熟練度に応じた差が明確に現れる技能であり,その評価や分析の対象として注目されている.
調理技能の評価では,ウェアラブルデバイスを活用して調理動作を詳細に記録し,分析を行える.
これにより,上達の過程や改善点を客観的に把握できるようになり,自己評価やモチベーションの向上が期待される.
例えば,現在の自分の動作を過去の記録と比較する際,調理技術が向上している様子を視覚的に確認でき,さらなる向上心を引き出せる.
また,熟練者との動作比較では,自身の動作における課題が明確になり,より効率的に技能向上を目指せると考えられる.

\section{目的とアプローチ}
本研究では,包丁を持った手の3次元的な動きをセンシングし,取得したデータを基にした特徴量抽出を目的とする.
本研究による提案手法は図\ref{fig:1}に示す.
\figimage{images/fig1.pdf}{150}{本研究の概要図}{fig:1}
アプローチとして,ウェアラブルセンサで取得した加速度データおよび角速度データを活用し,使用者の腕の動きを3次元的に推定する.
推定された動作データを詳細に分析し,切った回数や切るペースといった定量的な特徴量を抽出する.
この特徴量は,過去の自分との比較を通じた自己改善だけでなく,熟練者との比較による具体的な改善ポイントの把握にも役立つ.
これにより,使用者が調理技能を効果的に向上させるための支援ツールとしての活用が期待される.
% \section{関連研究}
ウェアラブルデバイスを用いた技能評価や技能向上に関する研究がある.
スマートフォン内蔵のセンサを使用して生活行動を推定する研究
\cite{携帯電話搭載センサによるリアルタイム生活行動認識システム}
ウェアラブルセンサを用いたヤスリがけ動作の技能評価による熟達者の動作の再現に関する研究,
\cite{ウェアラブルセンサを用いた熟練指導員のヤスリがけ技能主観評価値の再現}
,ウェアラブルセンサを用いた運転技能評価に関する研究
\cite{実世界に広がる装着型センサを用いた行動センシングとその応用:6. 装着型センサを用いた運転者行動センシング}
などがある.
これらの研究ではウェアラブルデバイスを用いて集めたデータをもとに技能評価を行い使用者の技術向上を目指している.
本研究では調理動作を対象する技能評価のための特徴量を抽出する.

調理行動の推定に専用の機材を用いる研究がある.
包丁に直接加速度センサを取り付け包丁技術を判定する研究\cite{加速度センサを用いた包丁技術向上支援システムの提案},マルチモーダルセンシングによる料理中
のマイクロ行動の認識を目指す研究\cite{マルチモーダルセンシングに基づく料理中のマイクロ行動認識の提案}
,加速度センサを腕に取り付けて調理動作の判定を行う研究\cite{手首装着型の加速度センサを用いた実時間調理行動認識手法の実現}がある.
これらの研究は主に調理機材やキッチン,人間にセンサ類を取り付けるなど,自作の装置を用いて調理動作を推定している.
また,ウェアラブルデバイスを用いた調理の切る動作の分析の研究がある\cite{kumazawaanalysis}.
しかし,この研究では分析の際に加速度しか使用していない.
本研究ではウェアラブルデバイスに内蔵されている加速度・角速度センサを用いて調理動作の推定を行う.
ウェアラブルデバイスを用いるとセンサ等を準備する必要がなくなり専門的な機材がなくてもデータの収集が可能である.


\section{ウェアラブルデバイスの回転量と移動距離の推定}
図\ref{fig:1}に本研究の全体像を示す.
各節では全体像内の「データの収集」「データの分析」「データの可視化」について述べる.

本研究ではウェアラブルデバイスとしてスマートウォッチを使用した.
1章で述べたようにウェアラブルデバイスの中にはセンサグローブやセンサリングなども存在している.
指先の動作まで取得可能なセンサグローブが理想である.しかし,一般的に普及しており料理という環境から防水性のあるスマートウォッチを採用した.
また,カメラを使わない理由としては,調理動作中の手の動きは立体的な動きであるため,複数視点から撮影を行う必要があるためである.

先行研究\cite{kumazawaanalysis}では加速度のみを用いて推定するものもあるが,推定できる動作がかなり限定的である.
そこで加速度・角速度を組み合わせて,端末の回転量や移動を算出し調理技能の分析に使用する.

\figimage{images/fig1.pdf}{80}{本研究の概要図}{fig:1}

\subsection{データの収集}
スマートウォッチでセンシングを行うアプリを自作し取得した加速度・角速度を使用して推定を行う.
基本的な包丁の握り方では図\ref{fig:2}のようにウォッチの向きと包丁の刃の向きが一致する.
\figimage{images/fig2.pdf}{80}{スマートウォッチと刃の向きの相関}{fig:2}
そこで収集するデータとしては,包丁を持つ側の手首の動きをセンシングしたものを収集する.
収集したデータは今後使用しやすいようにネットワークを介してオブジェクトストレージサーバーへ蓄積する仕組みとなっている.
\subsection{データの分析}
本節ではオブジェクトストレージサーバーから呼び出したデータを用いた端末の回転量と移動の推定について述べる.

今回は角速度と加速度にMadgwickフィルタを用いてセンサフージョンを行い回転量をクォータニオンとして導出している.
他の候補としてカルマンフィルタがあるが,モデルの構造が不明な場合に高精度を実現するのが難しい.
Madgwickフィルタは,事前に求めておくべきパラメータが少なく処理が高速な特徴がある.
そのため計算速度の観点からMadgwickフィルタを採用した.

クォータニオンとは回転の量を表すもので,オイラー角や回転行列と相互に変換可能であり,
上下がわからなくなるジンバルロックという現象が発生しない特徴がある.
求めた回転量を使用して重力加速度を導出し,加速度から重力加速度を取り除き線形加速度を導出している.
しかしこのままでは相対座標に基づいた移動なので,端末の姿勢を変化させながら動かすと異なる座標軸での移動と検出される.
そこで次の式を用いて変換を行う.
\[
\mathbf{a}_{\text{絶対}} = \mathbf{q} \cdot \mathbf{a}_{\text{相対}} \cdot \mathbf{q}^{-1}
\]
絶対座標へ変換すると端末の姿勢が変化した際にも同じ方向に対する移動として検出可能となる.
求めた絶対座標での線形加速度を二重積分して移動距離を導出している.
しかし,このままでは積分により図\ref{fig:3}のように誤差が蓄積しており正しい移動距離が推定できていない.
\figimage{images/fig3.pdf}{80}{積分による誤差と誤差の軽減}{fig:3}
そこでハイパスフィルタを用いて,積分誤差の軽減をしている.
角度については次の式をもとにクォータニオンから変換を行なっている.
    \[
	\text{Roll} = \tan^{-1}\left( \frac{2(w x + y z)}{1 - 2(x^2 + y^2)} \right)
	\]
	\[
	\text{Pitch} = \sin^{-1}\left( 2(w y - z x) \right)
	\]
	\[
	\text{Yaw} = \tan^{-1}\left( \frac{2(w z + x y)}{1 - 2(y^2 + z^2)} \right)
	\]
Roll・Pitch・YawがそれぞれX軸・Y軸・Z軸の角度を表している.
図\ref{fig:2}のように端末と包丁の動作は連動しているため,本推定手法により包丁の移動・角度を求められる.

\subsection{データの可視化}
分析されたデータを比較し,結果を表示する.過去の自分や他者との比較から調理技能の向上を目指す.
比較について,切る工程を例に述べる.特徴量として切った回数・
切るペース・包丁を入れた角度などがある.切った回数が他の人
より多い場合は切った物の幅が小さい可能性があり,逆に少な
い場合は幅が大きい可能性がある.切る
ペースが他より早い場合は手際が良い,逆に遅い場合は手際が悪いと
考えられる.包丁を入れた角度からは,切った対象の断面の面積に関係している可能性があり,切る際の重要な要素の一つであると考えられる.
このように複数のデータから調理技能の比較を行う.
利用者は比較結果から数値化された自身の調理技能や改善点がわ
かり,今後の調理技能の向上を促進する.結
果の表示方法については特徴量別に順位をつけるものや個別での
比較などが考えられる.今後他の方法がないかを検討する必要が
ある.

\section{評価実験}
本実験では3章で述べた特徴量抽出手法を刺身を切り分ける動作に対して行い,抽出した特徴量が調理技能の評価に利用できるかの確認を目的とする.
実験内容として刺身を切り分ける動作のセンシングを行う.
取得したデータをもとに3次元的な動きを推定し,特徴量の抽出を行う.
抽出した特徴量から調理技能の評価を行う.
本章では各節において実験の流れについて述べる.
\subsection{実験設定}
加速度と角速度を収集する自作アプリを起動したスマートウォッチ(PixelWatch2)を包丁を握る側の腕に装着しセンシングを行う.
被験者は10cmほどの魚の柵を刺身に切り分ける調理を行う.
刺身を切る動作は刃を入れる角度による断面の変化や包丁を引く回数から切り分けた身の厚さなどの複数の評価基準が得られる.

また,実際に切った区間のラベリングのために動画を撮影しながらデータ収集を行った.
後ほど切った区間のラベリングと推定した切っている区間が一致するかの確認に使用する.
\subsection{実験結果}
結果として得られた包丁の動作の一例を表したグラフを次の図\ref{fig:4}に示す.
\figimage{images/fig4.pdf}{80}{刺身を切る際の3次元的な動作のグラフ}{fig:4}
このデータは3章で述べた推定手法を用いて求めた端末の回転量から変換した角度と移動を表している.
誤差の軽減のためにハイパスフィルタを使用した.
赤くラベリングされている部分が刺身を切り分けている動作である.
また,青くラベリングされている部分は包丁を持つ動作と置く動作である.

3次元的な動きを扱うにあたりグラフで可視化しても分かりづらいため,手首の動きを画面上に再現するシステムを作成した.
作成した回転と移動を3次元空間での可視化を行うシステムが図\ref{fig:5}である.
\figimage{images/fig5.pdf}{80}{3次元空間での可視化システム}{fig:5}
この可視化システムによって,現実の動きとの比較が容易になり抽出しやすくなる.

切り分けている動作と準備動作のラベルは撮影した動画から手入力している.
動画との同期は3次元空間での可視化システムで見比べて行う.
特に変化が見られたY座標軸の移動距離と推定した切ったタイミングを表したグラフが図\ref{fig:6}である.

\figimage{images/fig6.pdf}{80}{推定結果のグラフ}{fig:6}

まずは切った回数の推定を行う.
動画によるラベリングとデータにより,切る際には一度奥(+)方向へ移動してから手前(-)方向へ移動しているとわかる.
そのため今回は極値検出を用いて切った回数の推定を行う.
切る際にメジャーを用い手首の移動距離を調べ,一番動かしていなかった切り込みを基準に閾値を2cmと定めた.
閾値以上となった極大値から次の極大値までの区間内で一番値が小さな極小値までで1回切ったと推定する.
その結果を図\ref{fig:6}で「切り始め」と「切り終わり」として表している.
結果から切ったと推定した部分とラベリングを行なった区間が一致した.
以上より切る動作を推定できた.
しかし包丁を持つ動作と置く動作も切っている部分と判定されてしまう場合があるため,あらかじめ包丁を持ってからセンシングを始めるなど改善する必要がある.
次に平均ペースを求める.
切るペースは一回毎の間の秒数を利用する.
また,その際のRoll(x)軸での平均角度も特徴量として使用する.

包丁を持つ動作と置く動作を除いたデータの特徴量を次の表\ref{table:1}に示す.
\begin{table}[ht]
    \centering
    \caption{特徴量}
    \label{table:1}
    \resizebox{70mm}{!}{ 
        \begin{tabular}{ccc}
            \hline\hline
            切った回数 & かかった時間(s) & 角度の平均(deg) \\
            \hline
            1 & 1.61 & -27.42 \\
            2 & 3.30 & -24.51 \\
            3 & 3.38 & -23.70 \\
            4 & 3.62 & -22.95 \\
            5 & 3.70 & -21.84 \\
            \hline
            全体の平均 & 3.12 & -24.08 \\
            \hline
        \end{tabular}
    }
\end{table}
表\ref{table:1}の結果から得られた特徴量は次のような評価の基準になる.
まず切った回数から刺身の厚さが算出できる.
今回のデータの場合5回切っているため6枚の刺身ができている.
切った柵が10cmなため10cmを6等分した結果一枚約1.6cmとわかる.
次に切るペースから手際の良さが算出できる.
今回の場合は平均ペースが3.12sとわかる.
最後に切る際の角度から切り方が推定できる.
今回の場合は柵に対して垂直な状態から右に24°ほど角度をつけて切っている.
そのため厚めに切る手法である平造りであると推定できる.
実験より本研究で抽出した特徴量は刺身を切る動作において調理技能の評価に利用できると考える.


\section{まとめと今後の展望}
本章では本論のまとめと今後の課題について述べる.
\subsection{まとめ}
%--------------------------------------------
\begin{thebibliography}{9}
    % \bibitem{携帯電話搭載センサによるリアル
    % タイム生活行動認識システム}
    % 大内 一成, 土井 美和子:携帯電話搭載センサによるリアル
    % タイム生活行動認識システム, 情報処理学会論文誌, Vol. 53,
    % No. 7, pp. 1675-1686 (2012).
    
    \bibitem{ウェアラブルセンサを用いた熟練指導員のヤスリがけ技能主観評価値の再現}
    榎堀 優ら:ウェアラブルセンサを用いた熟練指導員のヤスリがけ
    技能主観評価値の再現, 人工知能学会論文誌, Vol. 28, No. 4, pp. 391-399 (2013).
    
    % \bibitem{実世界に広がる装着型センサを用いた行動センシングとその応用:6. 装着型センサを用いた運転者行動センシング}
    % 多田 昌裕:実世界に広がる装着型センサを用いた行動センシングとその応用:6. 装着型センサを用いた運転者行動センシング, 情報処理, Vol. 54, No. 6, pp. 588-591 (2013).
    
    \bibitem{加速度センサを用いた包丁技術向上支援システムの提案}
    小林 花菜乃ら:
    加速度センサを用いた包丁技術向上支援システムの提案, DICOMO2020, Vol. 2020, pp. 1000-1003 (2020).
    
    % \bibitem{マ
    % ルチモーダルセンシングに基づく料理中のマイクロ行動認識
    % の提案}
    % 石山 時宗, 松井 智一, 藤本 まなと, 諏訪 博彦, 安本 慶一:マ
    % ルチモーダルセンシングに基づく料理中のマイクロ行動認識
    % の提案, 情報処理学会関西支部支部大会講演論文集, Vol. 2021, (2021).
    
    % \bibitem{手首装着型の加速度センサを用いた実時間調理行動認識手法の実現}
    % 大神 健司, 飛田 博章:手首装着型の加速度センサを用いた実時間調理行動認識手法の実現,
    %  人工知能学会全国大会論文集, Vol. 37, (2023).
    
    \bibitem{kumazawaanalysis}
    Ayato Kumazawa et al.:
    Analysis and Sharing of Cooking Actions Using Wearable Sensors, 
    \textit{IWIN}, Vol. 17, (2023).
     

\end{thebibliography}
\end{document}

%%% Local Variables: 
%%% mode: japanese-latex
%%% TeX-master: t
%%% End: 
