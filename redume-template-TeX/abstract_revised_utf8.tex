%
%  愛知工業大学 情報科学部 情報科学科
%    要旨集用LaTeXテンプレート(2016.11.28)
%
%  original by Nobuhiro Ito
%  revised by Susumu Suzuki & Masashi Morimoto on 2016.11.28
%
\documentclass{jarticle}
\pagestyle{empty}

%% レイアウト
\setlength{\topmargin}{-10.4mm}
\setlength{\headheight}{0mm}
\setlength{\headsep}{0mm}
\setlength{\textheight}{262mm}
\setlength{\textwidth}{180mm}
%\setlength{\topskip}{7mm}
\setlength{\evensidemargin}{-10.4mm} 
\setlength{\oddsidemargin}{-10.4mm} 
\setlength{\columnsep}{8mm}

%% パッケージ環境
% graphicx: 画像読み込み
%  dvipdfmxをドライバ指定することでpdf/png/jpeg形式の図を利用可能
%  異なるドライバを利用する場合はそのドライバ指定に変更
% \usepackage{graphicx} % suzuki
\usepackage[dvipdfmx]{graphicx} % suzuki
% subcaption: subfigureを置き換えたパッケージ(複数の図用)
% \setlength{\footskip}{12mm}
% \usepackage{subfigure}	 % suzuki
\usepackage{subcaption} % suzuki
% color:文章に色をつけたいとき
%  利用例:\textcolor{red}{文章}
% \usepackage{color}
\usepackage{booktabs}
\usepackage{setspace}

% \figimage コマンドの定義
%{画像のパス}{横幅(mm)}{キャプション}{ラベル名}
\newcommand{\figimage}[4]{
	\begin{figure}[tbh]
		\centering
		\includegraphics[width=#2 mm]{#1}
		\caption{#3}
		\label{#4}
	\end{figure}
}
% ---------------


% 行間調整
\setstretch{0.9}

%sectionのフォントサイズ修正
\makeatletter
\def\section{\@startsection {section}{1}{\z@}{2.5ex plus -1ex minus -.2ex}{1.3 ex plus .1ex}{\large\bf}}
\makeatother 

%subsectionのフォントサイズ修正
\makeatletter
\def\subsection{\@startsection {subsection}{1}{\z@}{1.5ex plus -1ex minus -.4ex}{0.3 ex plus .1ex}{\bf}}
\makeatother 

\begin{document}
\twocolumn[

\begin{center}
%タイトル
{\LARGE \textbf{手首の3次元的な動きの分析に基づく\\刺身を切る動作の特徴量抽出に関する研究}}\\
%サブタイトル
%{\Large \textbf{必要に応じてサブタイトル}}
\end{center}

\begin{center}
% 著者
\begin{tabular}{cccc}
% 1名の場合
%\multicolumn{4}{c}{K11001 愛工総和}\\
% 2名の場合
%& K11002 愛工七音 & X11003 愛工頼音 &\\
% 3名の場合
%K11001 愛工総和 & K11002 愛工今鹿 & X11003 愛工姫星&\\
% 4名の場合
%K11001 愛工総和 & K11002 愛工今鹿 & X11003 愛工姫星& X11004 愛工緑輝\\
\multicolumn{4}{c}{K21091 西幸希}\\
% 指導教員
\multicolumn{4}{c}{\textbf{指導教員} 梶克彦}
\end{tabular}
\hspace{2zw}
\end{center}
]

%--------------------------------------------
\section{はじめに}
近年ではスマートウォッチと
呼ばれる腕時計型のデバイスやセンサグローブと呼ばれる手袋型のデバイスなど動作を取得できるウェアラブルデバイスが普及し始めている.
それに伴い装着者の腕の動きを容易に取得可能となった.
取得したデータはさまざまな用途で使用されている.
例として,スポーツではアスリートの動きを詳細に計測してプロの動作と比較を行いパフォーマンスの向上や怪我の予防に役立てられている.
また,医療分野では,患者のリハビリテーションや日常生活動作を監視するためにウェアラブルデバイスが使用されている.
歩行解析や関節の可動域の測定を行い,回復状況の定量的に評価が可能となっている.
従来の主観的な評価に比べて,より客観的で精密な技能評価が可能になっており,リハビリプランの最適化や治療の効果測定に役立てられている.

スポーツや医療のように専門的な動作の技能評価は進んできているが,日常生活に関係してくる動作を評価しているものはまだ少ない.
日常動作の中で技能が必要となってくる動作の一つに調理があげられる.
調理技能の評価においても,これらのデバイスを用いて調理動作を詳細に記録・分析を行い,上達の過程や改善点を客観的に捉えられる.
例えば今と過去の自分を比較した場合,調理技術が向上しているのがわかりやすくなり,向上心が高まると考えられる.また熟練者と調理動作を比
較した場合も改善点が見つかり調理技能の向上に近づけると考えられる.

そこで本研究では包丁を持った手の3次元的な動きをセンシングしたデータを用いて算出し,特徴量の抽出を目的とする.
ウェアラブルセンサを用いて取得した加速度・角速度から使用者の腕の動きを3次元的に推定する.
推定した3次元の動きを分析し切った回数や切るペースなどを抽出する.
抽出した特徴量は過去の自分との比較や熟練者との比較により動作の改善を促進し,使用者の調理技能向上に使用できる.
% \section{関連研究}
本章では関連研究についてウェアラブルデバイスを用いた技能評価や技能向上・調理行動推定に分けて説明する.それぞれのカテゴリーが本研究との関わり
を述べる.

\subsection{ウェアラブルデバイスを用いた技能評価}
スマートフォン内蔵のセンサを使用して生活行動を推定する研究
\cite{携帯電話搭載センサによるリアルタイム生活行動認識システム}
やウェアラブルセンサを用いたヤスリがけ動作の技能評価による熟達者の動作の再現に関する研究
\cite{ウェアラブルセンサを用いた熟練指導員のヤスリがけ技能主観評価値の再現}
やウェアラブルセンサを用いた運転技能評価に関する研究
\cite{実世界に広がる装着型センサを用いた行動センシングとその応用:6. 装着型センサを用いた運転者行動センシング}
などがある.
これらの研究ではウェアラブルデバイスを用いて集めたデータをもとに技能評価を行うことで使用者の技術向上を目指している.
本研究では調理動作の技能評価による技能向上を目的としてウェアラブルセンサを使用する.

\subsection{調理行動推定の研究}
例えば調理行動の推定に専用の機材を用いる研究がある.
包丁に直接加速度センサを取り付け包丁技術を判定する研究\cite{加速度センサを用いた包丁技術向上支援システムの提案} やマルチモーダルセンシングによる料理中
のマイクロ行動の認識を目指す研究\cite{マルチモーダルセンシングに基づく料理中のマイクロ行動認識の提案}
や加速度センサを腕に取り付けて調理動作の判定を行う研究\cite{手首装着型の加速度センサを用いた実時間調理行動認識手法の実現}がある.
これらの研究は主に調理機材やキッチン,人間にセンサ類を取り付けるなどの自作の装置を用いて調理動作を推定している.
本研究ではウェアラブルセンサに内蔵されている加速度・角速度センサを用いて大掛かりな装置を使用せず調理工程の推定を行う.

\section{ウェアラブルセンサを用いた調理技能の分析}
図\ref{fig:1}に本研究の全体像を示す.本研究ではウェアラブルセンサとしてスマートウォッチを使用した.
先行研究\cite{kumazawaanalysis}では加速度のみを用いて推定するものもあるが,推定できる動作がかなり限定的である.
そこで加速度・角速度を組み合わせて,端末の回転量や移動を算出し調理技能の分析に使用する.

\figimage{images/fig1.pdf}{60}{本研究の概要図}{fig:1}

\subsection{データの収集}
スマートウォッチでセンシングを行うアプリを自作し取得した加速度・角速度を使用して推定を行う.
基本的な包丁の握り方だと図\ref{fig:2}のようにウォッチの向きと包丁の刃の向きが一致する.
\figimage{images/fig2.pdf}{60}{ウォッチと刃の向きの相関}{fig:2}
そこで収集するデータとしては,包丁を持つ側の手首の動きをセンシングしたものを使用して包丁の動きの推定を行う.
\subsection{データの加工}
センシングしたデータを用いて端末の回転量と移動の推定を行う.
今回は角速度と加速度にMadgwickフィルタを用いてセンサフージョンを行い回転量をクォータニオンとして導出している.
クォータニオンとは回転の量を表すもので,オイラー角や回転行列と相互に変換可能であり,
上下がわからなくなるジンバルロックという現象が発生しない特徴がある.
求めた回転量を使用して重力加速度を導出し,加速度から重力加速度を取り除き線形加速度を導出している.
しかしこのままでは相対座標に基づいた移動なので,端末の姿勢を変化させながら動かすと異なる座標軸での移動と検出される.
そこで次の式を用いて変換を行う.
\[
\mathbf{a}_{\text{絶対}} = \mathbf{q} \cdot \mathbf{a}_{\text{相対}} \cdot \mathbf{q}^{-1}
\]
絶対座標へ変換すると端末の姿勢が変化した際にも同じ方向に対する移動として検出可能となる.
求めた絶対座標での線形加速度を二重積分して移動距離を導出している.
しかし,このままでは積分により誤差が蓄積しており正しい移動距離が推定できていない.
そこでハイパスフィルタを用いて,積分誤差の軽減をしている.
3.2で述べたように端末と包丁の動作は連動しているため,本推定手法により包丁の移動・角度を求められる.



\section{評価実験}
本実験では3章で述べた特徴量抽出手法を刺身を切り分ける動作に対して行い,抽出した特徴量が調理技能の評価に利用できるかの確認を目的とする.
実験内容として刺身を切り分ける動作のセンシングを行う.
取得したデータをもとに3次元的な動きを推定し,特徴量の抽出を行う.
抽出した特徴量から調理技能の評価を行う.
本章では各節において実験の流れについて述べる.
\subsection{実験設定}
加速度と角速度を収集する自作アプリを起動したスマートウォッチ(PixelWatch2)を包丁を握る側の腕に装着しセンシングを行う.
被験者は10cmほどの魚の切り身を刺身に切り分ける調理を行う.
刺身を切る動作は刃を入れる角度による断面の変化や包丁を引く回数から切り分けた身の厚さなどの複数の評価基準が得られる.

また,実際に切った区間のラベリングのために動画を撮影しながらデータ収集を行った.
後ほど切った区間のラベリングと推定した切っている区間が一致するかの確認に使用する.
\subsection{実験結果}
結果として得られた包丁の動作の一例を表したグラフを次の図\ref{fig:4}に示す.
\figimage{images/fig4.pdf}{80}{刺身を切る際の3次元的な動作のグラフ}{fig:4}
このデータは3章で述べた推定手法を用いて求めた端末の回転量から変換した角度と移動を表している.
誤差の軽減のためにハイパスフィルタを使用した.
赤くラベリングされている部分が刺身を切り分けている動作である.
また,青くラベリングされている部分は包丁を持つ動作と置く動作である.

3次元的な動きを扱うにあたりグラフで可視化しても分かりずらいため,手首の動きを画面上に再現するシステムを作成した.
作成した回転と移動を3次元空間での可視化を行うシステムが図\ref{fig:5}である.
\figimage{images/fig5.pdf}{80}{3次元空間での可視化システム}{fig:5}
この可視化システムによって,現実の動きとの比較が容易になり特徴量の抽出を行いやすくなった.

切り分けている動作と準備動作のラベルは撮影した動画から手入力している.
動画との同期は3次元空間での可視化システムで見比べて行う.
特に変化が見られたY座標軸の移動距離と推定した切ったタイミングを表したグラフが図\ref{fig:6}である.

\figimage{images/fig6.pdf}{80}{推定結果のグラフ}{fig:6}

まずは切った回数の推定を行う.
動画によるラベリングとデータにより,切る際には一度奥(+)方向へ移動してから手前(-)方向へ移動しているとわかる.
そのため今回は極値検出を用いて切った回数の推定を行う.
切る際にメジャーを用い手首の移動距離を調べ,一番動かしていなかった切り込みを基準に閾値を2cmと定めた.
閾値以上となった極大値から次の極大値までの区間内で一番値が小さな極小値までで1回切ったと推定する.
その結果を図\ref{fig:6}で「切り始め」と「切り終わり」として表している.
結果から切ったと推定した部分とラベリングを行なった区間が一致した.
以上より切る動作を推定できた.
しかし包丁を持つ動作と置く動作も切っている部分と判定されてしまう場合があるため,あらかじめ包丁を持ってからセンシングを始めるなど改善する必要がある.
次に平均ペースを求める.
切るペースは1回1回の間の秒数を利用する.
また,その際のRoll(x)軸での平均角度も特徴量として使用する.

今回のデータの特徴量を包丁を持つ動作と置く動作を除きを表にしたものを次の表\ref{table:1}に示す.
\begin{table}[ht]
    \centering
    \caption{特徴量}
    \label{table:1}
    \resizebox{70mm}{!}{ 
        \begin{tabular}{ccc}
            \hline\hline
            切った回数 & かかった時間(s) & 角度の平均(deg) \\
            \hline
            1 & 1.61 & -27.42 \\
            2 & 3.30 & -24.51 \\
            3 & 3.38 & -23.70 \\
            4 & 3.62 & -22.95 \\
            5 & 3.70 & -21.84 \\
            \hline
            全体の平均 & 3.12 & -24.08 \\
            \hline
        \end{tabular}
    }
\end{table}
表\ref{table:1}の結果から得られた特徴量は次のような評価の基準になる.
まず切った回数から刺身の厚さが算出できる.
今回のデータの場合5回切っているため6枚の刺身ができている.
切った柵が10cmなので10cmを6等分した結果一枚約1.6cmと分厚めであるとわかる.
次に切るペースから手際の良さが算出できる.
今回の場合は平均ペースが3.12sと丁寧に切っているがあまり手際は良くないとわかる.
最後に切る際の角度から切り方の推定できる.
今回の場合は柵に対して垂直な状態から右に24°ほど角度をつけて切っている.
そのため厚めに切る手法である平造りであると推定できる.
実験より本研究で抽出した特徴量は刺身を切る動作において調理技能の評価に利用できると考える.


\section{おわりに}

本稿では,包丁を持った手の3次元的な動きをセンシングしたデータを用いて算出し特徴量の抽出を目的としている.
抽出した特徴量は過去の自分との比較や熟練者との比較により動作の改善を促進し,使用者の調理技能向上に使用する.
アプローチとしてウェアラブルセンサを用いて取得した加速度・角速度から使用者の腕の動きを3次元的に推定する.
推定した腕の移動・角度などの3次元的な動きを使用して刺身を切る動作における特徴量を抽出している.
抽出した特徴量は切った回数・切るペース・刃を入れた角度の3項目である.
抽出した特徴量をもとに刺身の厚さや手際の良さ,どの手法で切ったかなどの評価を行う.
本手法の評価実験を行い刺身を切る動作における特徴量を抽出した.
結果として動画から求めた切る区間と提案手法で求めた切った区間が一致した.
実験より本研究で抽出した特徴量は刺身を切る動作において調理技能の評価に利用できると考える.

今後の課題として特徴量をもとにした他者との比較や他の調理行動への応用,誤判定への対応の3つがあげられる.

1つ目については,今回の調査では特徴量を抽出したのみで他者との比較から調理技能の向上につながるかは検証できていない.
検証のためには他者との比較から調理に対する意識の変化など調理技能の向上と関係する変化が見られるか実験を行う必要がある.

2つ目の他の調理行動への応用については特徴量の抽出は切る動作しか行っていないため他の調理動作でも特徴量を算出できるかについて調査する必要がある.
炒める工程・混ぜる工程などがあるためそれぞれについて調査する.

最後に,包丁を持つ際などの動きが切っていると誤判定される問題がある.
そのため,センシング開始前に包丁をあらかじめ持っておくなどの対策を行う必要がある.
その際安全性にも考慮する必要があるため切り始める前と後に包丁を持った状態でも行える安全なジェスチャーを考える必要がある.

また今後の展望として特徴量を抽出する際に使用した3次元空間での可視化システムは結果のフィードバックへの応用を検討する.
抽出した特徴量のみでなく調理動作そのものを可視化して他者との比較を行う.
これにより特徴量を可視化するだけの場合と意識の変化に違いが見られるか検討できる.
%--------------------------------------------
\begin{thebibliography}{9}
    % \bibitem{携帯電話搭載センサによるリアル
    % タイム生活行動認識システム}
    % 大内 一成, 土井 美和子:携帯電話搭載センサによるリアル
    % タイム生活行動認識システム, 情報処理学会論文誌, Vol. 53,
    % No. 7, pp. 1675-1686 (2012).
    
    \bibitem{ウェアラブルセンサを用いた熟練指導員のヤスリがけ技能主観評価値の再現}
    榎堀 優ら:ウェアラブルセンサを用いた熟練指導員のヤスリがけ
    技能主観評価値の再現, 人工知能学会論文誌, Vol. 28, No. 4, pp. 391-399 (2013).
    
    % \bibitem{実世界に広がる装着型センサを用いた行動センシングとその応用:6. 装着型センサを用いた運転者行動センシング}
    % 多田 昌裕:実世界に広がる装着型センサを用いた行動センシングとその応用:6. 装着型センサを用いた運転者行動センシング, 情報処理, Vol. 54, No. 6, pp. 588-591 (2013).
    
    \bibitem{加速度センサを用いた包丁技術向上支援システムの提案}
    小林 花菜乃ら:
    加速度センサを用いた包丁技術向上支援システムの提案, DICOMO2020, Vol. 2020, pp. 1000-1003 (2020).
    
    % \bibitem{マ
    % ルチモーダルセンシングに基づく料理中のマイクロ行動認識
    % の提案}
    % 石山 時宗, 松井 智一, 藤本 まなと, 諏訪 博彦, 安本 慶一:マ
    % ルチモーダルセンシングに基づく料理中のマイクロ行動認識
    % の提案, 情報処理学会関西支部支部大会講演論文集, Vol. 2021, (2021).
    
    % \bibitem{手首装着型の加速度センサを用いた実時間調理行動認識手法の実現}
    % 大神 健司, 飛田 博章:手首装着型の加速度センサを用いた実時間調理行動認識手法の実現,
    %  人工知能学会全国大会論文集, Vol. 37, (2023).
    
    \bibitem{kumazawaanalysis}
    Ayato Kumazawa et al.:
    Analysis and Sharing of Cooking Actions Using Wearable Sensors, 
    \textit{IWIN}, Vol. 17, (2023).
     

\end{thebibliography}
\end{document}

%%% Local Variables: 
%%% mode: japanese-latex
%%% TeX-master: t
%%% End: 
