\section{評価実験}
% 本実験では2章で述べた特徴量抽出手法を刺身を切り分ける動作に対して行い,抽出した特徴量が調理技能の評価に利用できるかの確認を目的とする.
% 実験内容として刺身を切り分ける動作のセンシングを行う.
% 取得したデータをもとに3次元的な動きを推定し,特徴量の抽出を行う.
% 抽出した特徴量から調理技能の評価を行う.
% 本章では各節において実験の流れについて述べる.
\subsection{実験設定}
加速度と角速度を収集する自作アプリを起動したスマートウォッチ(PixelWatch2)を包丁を握る側の腕に装着しセンシングを行う.
被験者は10cmほどの魚の柵を刺身に切り分ける調理を切る手法を変えて2回行う.
柵は体に正対するように置く.
また,実際に切った区間のラベリングのために動画を撮影しながらデータ収集を行う.
後ほど切った区間のラベリングと推定した切っている区間が一致するかの確認に使用する.

本実験で抽出をする特徴量は切った回数・切るペース・刃の角度である.
切った回数が身の分厚さ,切るペースが手際の良さ,刃の角度が切り方の評価基準となる.
他にも特徴量として扱えるデータは存在するが,検証のためこの3項目について検討を行う.
\subsection{実験結果}
% 結果として得られた包丁の動作の一例として1回目のデータを表したグラフを次の図\ref{fig:4}に示す.
% このデータは3章で述べた推定手法を用いて求めた端末の回転量から変換した角度と移動を表している.
% 誤差の軽減のためにハイパスフィルタを使用した.
% 赤くラベリングされている部分が刺身を切り分けている動作である.
% また,青くラベリングされている部分は包丁を持つ動作と置く動作である.

3次元的な動きを扱うにあたりグラフで可視化しても分かりづらいため,手首の動きを画面上に再現するシステムを作成した.
作成した回転と移動を3次元空間での可視化を行うシステムが図\ref{fig:5}である.
\figimage{images/fig5.pdf}{80}{3次元空間での可視化システム}{fig:5}
この可視化システムによって,現実の動きとの比較が容易になり抽出しやすくなる.

切り分けている動作と準備動作のラベルは撮影した動画から手入力している.
動画との同期は3次元空間での可視化システムで見比べて行う.
特に変化が見られたY座標軸の移動距離と切ったタイミングの推定を表した一例のグラフが図\ref{fig:6}である.

\figimage{images/fig6.pdf}{80}{移動距離と切ったタイミングの推定結果グラフ}{fig:6}

動画によるラベリングとデータにより,切る際には一度奥(+)方向へ移動してから手前(-)方向へ移動しているとわかる.
そのため今回は極値検出を用いて切った回数の推定を行う.
切る際にメジャーを用いて手首の移動距離を調べ,一番動かしていなかった切り込みを基準に閾値を2cmと定めた.
閾値以上となった極大値から次の極大値までの区間内で一番値が小さな極小値までで1回切ったと推定する.
その結果を図\ref{fig:6}で「切り始め」と「切り終わり」として表している.
結果から切ったと推定した部分とラベリングを行った区間が一致した.
以上より,切る動作を推定できた.
しかし包丁を持つ動作と置く動作も切っている部分と判定されてしまう場合があるため,あらかじめ包丁を持ってからセンシングを始めるなど改善する必要がある.
次に平均ペースを求める.
切るペースは1回毎の間の秒数を利用する.
また,その際のRoll(x)軸での平均角度も特徴量として使用する.

包丁を持つ動作と置く動作を除いたデータの特徴量を表\ref{table:1}に示す.
平造りは切った回数と推定結果が一致したが,そぎ切りでは1回多く推定された.

% \begin{table}[ht]
%     \centering
%     \caption{平造りの特徴量}
%     \label{table:1}
%     \resizebox{65mm}{!}{ 
%         \begin{tabular}{ccc}
%             \hline\hline
%             切った回数 & かかった時間(s) & 角度の平均(deg) \\
%             \hline
%             1 & 1.61 & -27.42 \\
%             2 & 3.30 & -24.51 \\
%             3 & 3.38 & -23.70 \\
%             4 & 3.62 & -22.95 \\
%             5 & 3.70 & -21.84 \\
%             \hline
%             全体の平均 & 3.12 & -24.08 \\
%             \hline
%         \end{tabular}
%     }
% \end{table}

\begin{table}[ht]
    \centering
    \caption{各手法における特徴量}
    \label{table:1}
    \resizebox{80mm}{!}{% 表の幅を調整
        \renewcommand{\arraystretch}{2.5} % 行間を広げる
        \begin{tabular}{ccccc}
            \hline\hline
            \textbf{\huge 手法} & \textbf{\huge 切った回数} & \textbf{\huge 推定回数} & \textbf{\huge 平均ペース(s)} & \textbf{\huge 角度の平均(deg)} \\
            \hline
            \huge 平造り & \huge 5 & \huge 5 & \huge 3.12 & \huge -24.08 \\
            \huge そぎ切り & \huge 6 & \huge 7 & \huge 2.35 & \huge -44.22 \\
            \hline
        \end{tabular}
    }
\end{table}


表\ref{table:1}の結果から得られた特徴量は次のような評価の基準になる.
まず切った回数から刺身の厚さが算出できる.
平造りのデータの場合5回切っているため6枚の刺身ができている.
切った柵が10cmなため6等分した結果一枚約1.6cmとわかる.
% 表\ref{table:2}の場合10cmを8等分し約1.3cmとなる.
次に切るペースから手際の良さが算出できる.
平造りの場合は平均ペースが3.12秒と手際の評価に使用できる.
最後に切る際の角度から切り方が推定できる.
平造りの場合は柵に対して垂直な状態から右に24度ほど角度をつけて切っている.
そのため厚めに切る手法である平造りであると推定できる.
そぎ切りの場合だと平造りの角度からさらに右に20度ほど角度をつけて切っている.
そのため断面を広めに切る手法であるそぎ切りであると推定できる.

実験より本研究で抽出した特徴量は刺身を切る動作において調理技能の評価に利用できると考える.

