\chapter{評価実験}
本実験では3章で述べた特徴量抽出手法を刺身を切り分ける動作に対して行い,抽出した特徴量が調理技能の評価に利用できるかの確認を目的とする.
実験内容として刺身を切り分ける動作のセンシングを行う.
取得したデータをもとに3次元的な動きを推定し,特徴量の抽出を行う.
本章では各節において実験の流れについて述べる.
\section{実験設定}
加速度と角速度を収集する自作アプリケーションを起動したスマートウォッチ(PixelWatch2)を包丁を握る側の腕に装着しセンシングを行う.
柵は体に正対するように置き腕を柵に対して並行に置いた姿勢を初期姿勢とする図\ref{fig:homepose}.
\figimage{images/homepose.pdf}{150}{初期姿勢}{fig:homepose}
その際腕と並行な軸がRoll(X)軸,垂直な軸がPitch(Y)軸となっている.
被験者は刺身を切った経験がある学生1人と(被験者A),料理未経験の学生1人(被験者B),普段から調理をしている学生2人(被験者C・D)を対象とした.
被験者は10cmほどの魚の柵を刺身に切り分ける調理を平造り,そぎ切りと切る手法を変えて2回行う.
平造りは柵に対して刃を垂直に切る手法で,そぎ切りは刃を斜めに入れて切る手法である(図\ref{fig:cut}).
\figimage{images/cut.pdf}{150}{切る手法}{fig:cut}
刺身を切る動作は刃を入れる角度による断面の変化や包丁を引く回数から切り分けた身の厚さなどの複数の評価基準が得られる.

また,実際に切った区間のラベリングのために動画を撮影しながらデータ収集を行った.
後ほど切った区間のラベリングと推定した切っている区間が一致するかの確認に使用する.
\section{実験結果}
結果として得られた包丁の動作の一例として1回目のデータを表したグラフを次の図\ref{fig:4}に示す.
\figimage{images/fig4.pdf}{150}{刺身を切る際の3次元的な動作のグラフ}{fig:4}
このデータは3章で述べた推定手法を用いて求めた端末の回転量から変換した角度と移動を表している.
誤差の軽減のためにハイパスフィルタを使用した.
赤くラベリングされている部分が刺身を切り分けている動作である.
また,青くラベリングされている部分は包丁を持つ動作と置く動作である.

4章で作成した手首の動きを画面上に再現するシステムを使用して比較を行った様子が図\ref{fig:5}である.
\figimage{images/fig5.pdf}{150}{可視化システムと現実の比較}{fig:5}
この可視化システムによって,現実の動きとの比較が容易になり抽出しやすくなる.

切り分けている動作と準備動作のラベルは撮影した動画から手入力している.
動画との同期は3次元空間での可視化システムで見比べて行う.
特に変化が見られたY座標軸の移動距離と切ったタイミングの推定を表したグラフが図\ref{fig:6}である.

\figimage{images/fig6.pdf}{150}{推定結果のグラフ}{fig:6}

まずは切った回数の推定を行う.
動画によるラベリングとデータにより,切る際には一度奥(+)方向へ移動してから手前(-)方向へ移動しているとわかる.
そのため今回は極値検出を用いて切った回数の推定を行う.
切る際にメジャーを用いて手首の移動距離を調べ,一番動かしていなかった切り込みを基準に閾値を2cmと定めた.
閾値以上となった極大値から次の極大値までの区間内で一番値が小さな極小値までで1回切ったと推定する.
その結果を図\ref{fig:6}で「切り始め」と「切り終わり」として表している.
結果から切ったと推定した部分とラベリングを行った区間が一致した.
以上より,切る動作を推定できた.
しかし包丁を持つ動作と置く動作も切っている部分と判定されてしまう場合があるため,あらかじめ包丁を持ってからセンシングを始めるなど改善する必要がある.
次に平均ペースを求める.
切るペースは1回毎の間の秒数を利用する.
また,その際のRoll(x)軸での平均角度も特徴量として使用する.


%//表帰る
包丁を持つ動作と置く動作を除いたデータの特徴量を表\ref{table:1},表\ref{table:2}に示す.
表\ref{table:1}が平造り,表\ref{table:2}がそぎ切りの結果である.
しかし,未経験者である被験者Bからはうまく特徴量を抽出できなかった.
\begin{table}[ht]
    \centering
    \caption{平造りの特徴量}
    \label{table:1}
    \resizebox{100mm}{!}{ 
        \begin{tabular}{ccccc}
            \hline\hline
            被験者 & 推定回数 & 実際の回数 & 平均ペース(s) & 角度の平均(deg) \\
            \hline
            A & 5 & 5 & 3.12 & -24.08 \\
            B & 6 & 9 & 14.12 & -47.38 \\
            C & 6 & 7 & 4.13 & -34.58 \\
            D & 6 & 4 & 3.97 & -35.94 \\
            \hline
        \end{tabular}
    }
\end{table}

\begin{table}[ht]
    \centering
    \caption{そぎ切りの特徴量}
    \label{table:2}
    \resizebox{100mm}{!}{ 
        \begin{tabular}{ccccc}
            \hline\hline
            被験者 & 推定回数 & 実際の回数 & 平均ペース(s) & 角度の平均(deg) \\
            \hline
            A & 7 & 6 & 2.35 & -44.22 \\
            B & 6 & 12 & 4.13 &  -56.08 \\
            C & 8 & 9 & 3.31 & -48.01 \\
            D & 6 & 6 & 3.87 & -47.64 \\
            \hline
        \end{tabular}
    }
\end{table}
表\ref{table:1},表\ref{table:2}の結果から得られた特徴量は次のような評価の基準になる.
まず切った回数から刺身の厚さが算出できる.
表\ref{table:1}の被験者Aのデータの場合5回切っているため6枚の刺身ができている.
切った柵が10cmなため10cmを6等分した結果一枚約1.6cmとわかる.
また回数の推定精度は約8割(被験者Bを除く)となり,改善の余地があるが十分な精度が得られた.
次に切るペースから手際の良さが算出できる.
平均ペースの比較を行い今回の実験で一番手際が良かったのは被験者Aとなる.
表\ref{table:1},表\ref{table:2}の角度を比較し,どの被験者も平造りよりそぎ切りの方が10から20度ほど角度をつけて切っていると判明した.
そのため,平造りの角度を基準とした場合刃の角度から切る手法が推定できると考える.

今回の実験では未経験者の被験者Bの推定がうまくいかない結果となった.
被験者Bの推定結果のグラフを次の図\ref{fig:heta}に示す.
\figimage{images/heta.pdf}{150}{問題があった結果のグラフ}{fig:heta}
グラフを見たところ,移動距離の推定が上手く行っておらず,切る回数の判定ができていないと判明した.

被験者Bの推定がうまくいかなかった理由として,以下のような内容が考察できる.
動画を見返し異なる点を探した際に,切り方が他の3人と異なると判明した.
次の図\ref{fig:good_cut}に問題がなかった被験者の切り方,図\ref{fig:bad_cut}に問題があった被験者の切り方を示す.
\figimage{images/good_cut.pdf}{150}{問題がなかった被験者の切り方}{fig:good_cut}
\figimage{images/bad_cut.pdf}{150}{問題があった被験者の切り方}{fig:bad_cut}
他の被験者は包丁の刃元から1回の引きで切っているが,被験者Bのみ包丁の刃先で押しながら小刻みに切っていると判明した.
小刻みに動かすような変化量が小さな動作があるため,移動距離推定の際に行うフィルタリングによって変化が消されてしまったと考えられる.
そのため,積分時に変化量が与えられなかった区間が発生し結果が想定したものではなくなったと考えられる.
また,手前から押して切っていたため奥から手前に切り込む想定をしていた本手法ではうまく切った判定が結果が行われなかったと考えられる.
