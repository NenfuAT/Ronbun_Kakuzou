\chapter{推定システム}
本章では3章で述べた推定手法を用いて作成したシステムについて述べる.
本システムのシステム構成は次の図\ref{fig:sys}の通りである.
\figimage{images/system.pdf}{150}{システム構成図}{fig:sys}
まずウェアラブルデバイスを用いてセンシングを行いストレージサーバへ保存しておく.
次に可視化システムから推定を行いたいセンサデータを選択し,推定サーバへデータ保存先のURLが送られる.
推定サーバは送られてきたURLからセンサデータを取得し計算を行った後結果を可視化システムへ返却する.
返却された推定結果を元に手首の動きを画面上に再現し特徴量の抽出に役立てる流れを行なっている.
各節では本システムを「センサデータ取得アプリ」「推定サーバ」「可視化システム」に分けて説明する.
\section{センサデータ取得アプリ}
スマートウォッチを用いたセンシングには以下の3つの問題がある.
初めにスマートウォッチはスマートフォンと異なりケーブルを用いたデータの送信が不可能である.
スマートフォンはUSBケーブルを用いてPCとのデータ送信が可能だがスマートウォッチにはUSB端子が存在していない.
次に,内部ストレージの容量もスマートフォンより劣っている.
センシングしたデータをスマートウォッチに保持し続けた場合内部ストレージの容量がいずれ足りなくなり新しくデータの取得が不可能となってしまう.
最後に加速度・角速度を同時に保存できるようなアプリはストアに公開されていないという問題がある.
単一のデータを取得するようなアプリは公開されていたが複数のデータを取得するアプリは見つからなかった.
また,存在していたとしても上の2つの問題も解決する必要がある.

以上の問題を解決するために
加速度・角速度センサを同時に取得可能で,データの送信のためサーバとの通信も行えるアプリをKotlinを用いて自作した.
作成したアプリが次の図\ref{fig:sensing}である.
\figimage{images/sensing.pdf}{150}{センシングアプリの各画面}{fig:sensing}
機能としては複数のセンサを選択して同時にセンシングを行う機能とデータをサーバへ送信する機能がある.
データの送信先としてオブジェクトストレージのAmazonS3互換オープンソースであるMiniOを使用したサーバを作成した.
オブジェクトストレージサーバを作成した際に一緒に作成したAPIを使用し,センシングアプリからオブジェクトストレージサーバへデータをcsvファイルとして保存している.
その際送信に成功したタイミングでスマートウォッチ内のデータは削除することで内部ストレージの問題も解決した.



\section{推定サーバ}
3章で述べた推定手法をAPIサーバとして実装したものが推定サーバである.
クォータニオンの計算など複雑な計算を行う必要があるため,計算系のライブラリが充実しているPythonを使用してサーバを作成した.
可視化システムから受け取った署名付きURLを用いてMiniOサーバからセンサデータのcsvファイルを取得している.
取得したデータはタイムスタンプが一致していない場合があるので線形補間を用いて加速度・角速度のデータ数・タイムスタンプを一致させる.
補完後のデータを用いて3章で述べた推定手法を使用し,計算後のクォータニオンと移動距離をcsvファイルとして可視化システムへ返却している.

\section{可視化システム}
ここまでの手法で求めた回転・移動は3次元の動きのためグラフで可視化することが困難であり特徴量の抽出が難しい.
そこで手首の動きを画面上に再現するシステムを作成した.