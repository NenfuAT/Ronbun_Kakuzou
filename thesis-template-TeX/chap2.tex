\chapter{関連研究}
\section{技能評価に関する研究}
大半の行動推定の研究ではセンサを用いてデータ収集しているがセンサを使わず,インターネット上のビッグデータを用いた方法\cite{マイクロブログにおけるユーザの属性と習慣行動の推定に関する研究}もある.
しかし,実際の動作に基づいた評価を行いたい本研究では向いていない.
大半の研究で使われるセンサにはさまざまな種類がある.
例えば加速度センサを用いた方法がある.加速度センサを活用した非装着型の人間の行動推定システム\cite{加速度センサを活用した非装着型の人間の行動推定システム}や単一3軸加速度センサによる行動推定に関する研究\cite{単一3軸加速度センサによる行動推定},加速度センサから収集した人間行動データのサンプリング周波数推定手法に関する研究\cite{加速度センサから収集した人間行動データのサンプリング周波数推定手法}などがある.
他にもセンサとして磁気や温度・湿度を用いた小型センサモジュール搭載シューズを用いた行動センシングの研究\cite{小型センサモジュール搭載シューズを用いた行動センシング}
,マイクロフォンを用いた携帯電話搭載センサによるリアルタイム生活行動認識システムに関する研究\cite{携帯電話搭載センサによるリアルタイム生活行動認識システム},
気圧を用いた気圧センシング技術を用いた行動認識手法の研究\cite{気圧センシング技術を用いた行動認識手法},
Wi-Fiを用いたWiFiバックスキャッタータグを用いた非接触生活行動認識システムの研究\cite{WiFiバックスキャッタータグを用いた非接触生活行動認識システムの提案},
BLEを用いたヘルスケアにおける活動認識に関する研究\cite{Multimodal Wearable Sensing for Fine-Grained Activity Recognition in Healthcare},
カメラ映像を用いた人間の姿勢推定の研究\cite{GLPose: Global-Local Representation Learning for Human Pose Estimation},
3次元点群を用いた3次元点群を用いたマイクロ行動認識手法の研究\cite{3次元点群を用いたマイクロ行動認識手法の提案},
照度センサを用いた在宅時の活動見守りシステムの研究\cite{照度センサを用いた在宅時の活動見守りシステム}などがある.
さらに複数のセンサを組み合わせる方法として加速度センサに加えて角速度センサが内蔵された慣性センサ(IMU:Inertial Measurement Unit)を用いる方法がある\cite{Human Action Recognition Method Based on Wearable Inertial Sensor}\cite{Real-Time Joint Axes Estimation of the Hip and Knee Joint during Gait Using Inertial Sensors}\cite{歩行者自律測位における行動センシング知識の利用}.
センシング方法には大きく分けてセンサを場所に固定する必要のある設置型と人が持ち歩く・付ける機器だけで完結する装着型の2種類がある.
設置型の場合キッチンの環境などに依存してしまうため本研究には向いていない.
そのため本研究では装着型のセンサを用いたセンシングを行う.

装着型センサを用いた技能評価や行動推定に関する研究は,近年多くの注目を集めておりさまざまな分野で実施されている.
例えば,スマートフォン内蔵のセンサを使用して生活行動を推定する研究が進められており,リアルタイムで生活行動を認識するシステムが提案されている\cite{携帯電話搭載センサによるリアルタイム生活行動認識システム}.
これにより,日常生活における行動のモニタリングや分析が可能となる.
個々の行動パターンに基づいたアシスト技術の開発や,健康管理の支援が期待されている.
また,ウェアラブルセンサを用いた技能評価の研究も広がりを見せている.
例えば,ヤスリがけ動作に関する研究では,熟練者の動作パターンをウェアラブルセンサで測定したデータをもとに技能の評価を行い,技術の再現を目指した研究が行われている\cite{ウェアラブルセンサを用いた熟練指導員のヤスリがけ技能主観評価値の再現}.
さらに,運転技能の評価に関する研究では運転中の行動のセンシングを行い,ドライバーの運転技術や安全性を評価し,運転行動の改善に向けたフィードバックを提供する手法が提案されている\cite{実世界に広がる装着型センサを用いた行動センシングとその応用:6. 装着型センサを用いた運転者行動センシング}\cite{装着型加速度センサを用いた運転中の行動推定}.
これらの研究では,装着型センサから得られるデータの解析を行い,熟練度や技能の向上を目指している.
本研究では,調理動作を対象にした技能評価を行う.
調理行動は複雑で多様な動作を含んでおり,その動作の正確な評価は,調理技能の向上や効率的な調理技術の指導に不可欠である.
ウェアラブルセンサを使用して,調理動作の特徴を抽出し,技術の評価に役立つ指標の導出を目指す.
このアプローチにより,従来の主観的な評価方法に代わって,定量的かつ客観的な技能評価を実現し,調理技能の向上に貢献できると考えられる.

\section{調理行動に関する研究}
調理行動の推定に関する研究は多様なアプローチが取られており,専用の機材を用いる研究が数多く存在する.
例えば,包丁に直接加速度センサを取り付ける手法では,包丁の動きを詳細に捉え切る技術の判定や分析が行われている.この手法では,加速度のデータに機械学習を行い調理技術の評価に活用されている\cite{加速度センサを用いた包丁技術向上支援システムの提案}.
また,マルチモーダルセンシングを用いた研究では複数のセンサデータを組み合わせ,料理中の微細な行動を認識する試みが進められている.
この研究では,動作データと音声,さらに画像データを組み合わせ,行動の正確な認識を可能にしている\cite{マルチモーダルセンシングに基づく料理中のマイクロ行動認識の提案}.
さらに,カメラを用いた調理行動の分析研究では,画像処理技術を駆使して,手の形状や動作軌跡を解析し,調理プロセス全体を記録するシステムが提案されている\cite{Cooking Activities Recognition in Egocentric Videos Using Hand Shape Feature with Openpose}.
一方,匂いセンサを用いた研究では,空気中の揮発性成分をセンシングし,揚げ物を検出する手法が開発されている.
この手法は,揚げ物の進行状況や適切な調理時間をモニタリングするために利用されており,キッチン環境の自動化にも応用可能である\cite{NOSE: A Novel Odor Sensing Engine for Ambient Monitoring of the Frying Cooking Method in Kitchen Environments}.
さらに,加速度データを利用した調理工程の分析研究では,調理の各ステップを時系列データとして取得し,その特徴を基に工程を分類するシステムが構築されている.
この研究では,調理者の動作特性を学習し,調理技術の上達過程を評価する試みが行われている\cite{A dataset for complex activity recognition with micro and macro activities in a cooking scenario}.また,加速度センサを腕に装着して調理動作をリアルタイムで判定する研究では,簡単な装置でありながら高い精度を維持している点が特徴である\cite{手首装着型の加速度センサを用いた実時間調理行動認識手法の実現}.
これらの研究は,調理機材やキッチン環境にセンサを設置したり,調理者に特定の装置を装着するなど,専用の機材を用いるアプローチが一般的である.
しかし,このような方法では,導入時のコストや装置の準備に時間がかかる場合が多く,一般家庭での日常的な利用には課題がある.
本研究では,ウェアラブルデバイスに内蔵されている加速度センサおよび角速度センサを活用し,これらの課題に対応する.
ウェアラブルデバイスは,日常的に装着される前提で設計されており,追加のセンサを準備する必要がないため,導入コストを削減しつつデータ収集を容易にする.
このようなデバイスを利用すると,家庭内での調理行動の記録や分析が可能となり,多くの人々が調理技術の向上に取り組める環境を提供できると期待される.

\section{先行研究}
本研究はお料理センシングというプロジェクトの一環として行われている.
お料理センシングプロジェクトの全体像は次の図\ref{fig:project}である.
\figimage{images/project.pdf}{150}{プロジェクトの概要図}{fig:project}
本研究では図\ref{fig:project}における左下の特徴量データの抽出部分を担当している.

同プロジェクトの先行研究として,ウェアラブルデバイスを活用した調理における切る動作の分析が行われている\cite{kumazawaanalysis}.
この研究では,加速度データのみを用いて分析が実施されている.
具体的には,包丁とまな板が接触した際に生じる加速度を検知し,切った回数をカウントする手法が採用されている.

しかしながら,この手法にはいくつかの課題が存在する.
例えば,きゅうりを切るような明確な動作には問題なく対応可能である一方で,包丁とまな板が勢いよく衝突しない静かな動作では,加速度の変化が小さく検知が難しい.
さらに,切った回数以外の特徴を取得するには限界がある.
特に,切る際の包丁の角度や動作のスムーズさといった情報は,加速度データだけでは算出できない.

これらの課題を解決するために,本研究では加速度センサと角速度センサを組み合わせたセンサフュージョン手法を導入する.
このアプローチにより,切った回数の計測だけでなく,包丁の角度や動作の詳細な特徴を含む多様なデータの抽出が可能となり,より高精度な動作分析の実現を目指す.

