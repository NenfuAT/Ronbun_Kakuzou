\chapter{関連研究}
\section{技能評価に関する研究}
大半の行動推定の研究ではセンサを用いてデータ収集しているがセンサを使わない方法\cite{マイクロブログにおけるユーザの属性と習慣行動の推定に関する研究}もある.
しかし,実際の動作に基づいた評価を行いたい本研究では向いていない.
大半の研究で使われるセンサにはさまざまな種類がある.
例えば加速度センサを用いた方法がある\cite{加速度センサを活用した非装着型の人間の行動推定システム}\cite{単一3軸加速度センサによる行動推定}\cite{加速度センサから収集した人間行動データのサンプリング周波数推定手法}.
他にもセンサとして磁気や温度・湿度\cite{小型センサモジュール搭載シューズを用いた行動センシング}マイクロフォン\cite{携帯電話搭載センサによるリアルタイム生活行動認識システム},
気圧\cite{気圧センシング技術を用いた行動認識手法},
Wi-Fi\cite{WiFiバックスキャッタータグを用いた非接触生活行動認識システムの提案},
BLE\cite{Multimodal Wearable Sensing for Fine-Grained Activity Recognition in Healthcare},
カメラ映像\cite{GLPose: Global-Local Representation Learning for Human Pose Estimation},
3次元点群\cite{3次元点群を用いたマイクロ行動認識手法の提案},
照度\cite{照度センサを用いた在宅時の活動見守りシステム}などがある.
さらに複数のセンサを組み合わせる方法として加速度センサに加えて角速度センサが内蔵された慣性センサ(IMU:Inertial Measurement Unit)を用いる方法がある\cite{Human Action Recognition Method Based on Wearable Inertial Sensor}\cite{Real-Time Joint Axes Estimation of the Hip and Knee Joint during Gait Using Inertial Sensors}\cite{歩行者自律測位における行動センシング知識の利用}.
センシング方法には大きく分けてセンサを場所に固定する必要のある設置型と人が持ち歩く・付ける機器だけで完結する装着型の2種類がある.
設置型の場合キッチンの環境などに依存してしまうため本研究には向いていない.
そのため本研究では装着型のセンサを用いたセンシングを行う

装着型センサを用いた技能評価・行動推定に関する研究がある.
スマートフォン内蔵のセンサを使用して生活行動を推定する研究
\cite{携帯電話搭載センサによるリアルタイム生活行動認識システム}
ウェアラブルセンサを用いたヤスリがけ動作の技能評価による熟達者の動作の再現に関する研究,
\cite{ウェアラブルセンサを用いた熟練指導員のヤスリがけ技能主観評価値の再現}
,ウェアラブルセンサを用いた運転技能評価に関する研究
\cite{実世界に広がる装着型センサを用いた行動センシングとその応用:6. 装着型センサを用いた運転者行動センシング}
\cite{装着型加速度センサを用いた運転中の行動推定}
などがある.
これらの研究では装着型センサによるセンシングで集めたデータをもとに技能評価を行い使用者の技術向上を目指している.
本研究では調理動作を対象とする技能評価のための特徴量をセンサデータから抽出する.

\section{調理行動に関する研究}
調理行動の推定に専用の機材を用いる研究がある.
包丁に直接加速度センサを取り付け包丁技術を判定する研究\cite{加速度センサを用いた包丁技術向上支援システムの提案},マルチモーダルセンシングによる料理中
のマイクロ行動の認識を目指す研究\cite{マルチモーダルセンシングに基づく料理中のマイクロ行動認識の提案}
,カメラを用いた調理行動の分析に関する研究\cite{Cooking Activities Recognition in Egocentric Videos Using Hand Shape Feature with Openpose}
,空気をセンシングし揚げ物を検出する研究\cite{NOSE: A Novel Odor Sensing Engine for Ambient Monitoring of the Frying Cooking Method in Kitchen Environments}
,加速度を利用して調理工程を分析する研究\cite{A dataset for complex activity recognition with micro and macro activities in a cooking scenario} 
加速度センサを腕に取り付けて調理動作の判定を行う研究\cite{手首装着型の加速度センサを用いた実時間調理行動認識手法の実現}がある.
これらの研究は主に調理機材やキッチン,人間にセンサ類を取り付けるなど,自作の装置を用いて調理動作を推定している.
また,ウェアラブルデバイスを用いた調理の切る動作の分析の研究がある\cite{kumazawaanalysis}.
しかし,この研究では分析の際に加速度しか使用していない.
本研究ではウェアラブルデバイスに内蔵されている加速度・角速度センサを用いて調理動作の推定を行う.
ウェアラブルデバイスを用いるとセンサ等を準備する必要がなくなり専門的な機材がなくてもデータの収集が可能である.

