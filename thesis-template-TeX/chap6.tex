\chapter{おわりに}
\section{まとめ}
本稿では,包丁を持った手の3次元的な動きをセンシングしたデータを用いて算出し特徴量の抽出を目的としている.
抽出した特徴量は過去の自分との比較や熟練者との比較により動作の改善を促進し,使用者の調理技能向上に使用する.
アプローチとしてウェアラブルセンサを用いて取得した加速度・角速度から使用者の腕の動きを3次元的に推定する.
推定した腕の移動・角度などの3次元的な動きを使用して刺身を切る動作における特徴量を抽出している.
抽出した特徴量は切った回数・切るペース・刃を入れた角度の3項目である.
抽出した特徴量をもとに刺身の厚さや手際の良さ,どの手法で切ったかなどの評価を行う.
本手法の評価実験を行い刺身を切る動作における特徴量を抽出した.
結果として動画から求めた切る区間と提案手法で求めた切った区間が一致した.
実験より本研究で抽出した特徴量は刺身を切る動作において調理技能の評価に利用できると考える.

\section{今後の課題}
今後の課題として特徴量をもとにした他者との比較や他の調理行動への応用,誤判定への対応の3つがあげられる.

1つ目については,今回の調査では特徴量を抽出したのみで他者との比較から調理技能の向上につながるかは検証できていない.
検証のためには他者との比較から調理に対する意識の変化など調理技能の向上と関係する変化が見られるか実験を行う必要がある.

2つ目の他の調理行動への応用については特徴量の抽出は切る動作しか行っていないため他の調理動作でも特徴量を算出できるかについて調査する必要がある.
炒める工程・混ぜる工程などがあるためそれぞれについて調査する.

最後に,包丁を持つ際などの動きが切っていると誤判定される問題がある.
そのため,センシング開始前に包丁をあらかじめ持っておくなどの対策を行う必要がある.
その際安全性にも考慮する必要があるため切り始める前と後に包丁を持った状態でも行える安全なジェスチャーを考える必要がある.

また今後の展望として特徴量を抽出する際に使用した3次元空間での可視化システムは結果のフィードバックへの応用を検討する.
抽出した特徴量のみでなく調理動作そのものを可視化して他者との比較を行う.
これにより特徴量を可視化するだけの場合と意識の変化に違いが見られるか検討できる.