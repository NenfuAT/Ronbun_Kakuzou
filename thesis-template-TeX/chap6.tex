\chapter{おわりに}
\section{まとめ}
本稿では,包丁を持った手の3次元的な動きをセンシングしたデータを用いて算出し特徴量の抽出を目的としている.
抽出した特徴量は過去の自分との比較や熟練者との比較により動作の改善を促進し,使用者の調理技能向上に使用する.
アプローチとしてウェアラブルセンサを用いて取得した加速度・角速度から使用者の腕の動きを3次元的に推定する.
推定した腕の移動・角度などの3次元的な動きを使用して刺身を切る動作における特徴量抽出手法の提案を行なう.
提案手法を実現するために加速度・角速度を同時にセンシングするためのWearOSアプリケーションを作成した.
提案した特徴量抽出手法を用いた推定システムの作成と推定結果の可視化のためのシステムを作成した.
抽出した特徴量は切った回数・切るペース・刃を入れた角度の3項目である.
抽出した特徴量をもとに刺身の厚さや手際の良さ,どの手法で切ったかなどの評価を行う.
本手法の評価実験を行い刺身を切る動作における特徴量を抽出した.
結果として動画から求めた切る区間と提案手法で求めた切った区間が一致した.
実験より本研究で抽出した特徴量は刺身を切る動作において調理技能の評価に利用できると考える.

% \section{今後の課題}
% 今後の課題として特徴量をもとにした他者との比較や他の調理行動への応用,誤判定への対応の3つがあげられる.

% 1つ目については,今回の調査では特徴量を抽出したのみで他者との比較から調理技能の向上につながるかは検証できていない.
% 検証のためには他者との比較から調理に対する意識の変化など調理技能の向上と関係する変化が見られるか実験を行う必要がある.

% 2つ目の他の調理行動への応用については特徴量の抽出は切る動作しか行っていないため他の調理動作でも特徴量を算出できるかについて調査する必要がある.
% 炒める工程・混ぜる工程などがあるためそれぞれについて調査する.

% 最後に,包丁を持つ際などの動きが切っていると誤判定される問題がある.
% そのため,センシング開始前に包丁をあらかじめ持っておくなどの対策を行う必要がある.
% その際安全性にも考慮する必要があるため切り始める前と後に包丁を持った状態でも行える安全なジェスチャーを考える必要がある.

% また今後の展望として特徴量を抽出する際に使用した3次元空間での可視化システムは結果のフィードバックへの応用を検討する.
% 抽出した特徴量のみでなく調理動作そのものを可視化して他者との比較を行う.
% これにより特徴量を可視化するだけの場合と意識の変化に違いが見られるか検討できる.

\section{今後の課題}
本研究における今後の課題として,以下の4点が挙げられる.
特徴量をもとにした他者との比較,刺身における残りの特徴量の抽出,他の調理行動への応用,そして誤判定への対応である.
それぞれについて詳しく説明する.

まず1つ目の特徴量をもとにした他者との比較を行う必要がある.
本研究では調理行動に関連する特徴量の抽出に焦点を当てたが,他者との比較を通じて調理技能の向上にどのような影響を与えるかについては検証が行われていない.
この課題に取り組むためには,複数の被験者を対象とした実験を実施し,他者との比較を通じて調理に対する意識の変化やモチベーションの向上が見られるかを検討する必要がある.
例えば,特定の特徴量を向上させる訓練を行ったグループと,そうでないグループの間で調理技能の変化を比較し,その効果を明らかにできる可能性がある.
さらに,意識の変化が行動に与える影響についても調べる必要がある.

2つ目に,刺身における残りの特徴量を抽出する必要がある.
今回抽出した特徴量の他に取得できるものが存在しているため,抽出する手法を考案する必要がある.
また,今回抽出したものの中でも小刻みに動かしたかどうかが腕の動きをもとに取得ができなかった.
そのため,切り込み判定中の生の加速度の分析から小刻みに動かしたかを判定するなど,別のアプローチを考案する必要がある.


3つ目に,他の調理行動への応用を行う必要がある.
本研究では主に切る動作における特徴量の抽出を行ったが,調理行動には他にも多様な工程が存在する.
例えば,炒める,混ぜる,こねるといった動作が挙げられる.
これらの動作についても,切る動作と同様に特徴量を算出し,それが調理技能や行動分析に役立つかを調査する必要がある.
それぞれの動作において,使用される道具や力の加え方が異なるため,動作特有の特徴量を抽出する手法の確立が課題となる.
また,これらの動作ごとの特徴量を統合し,調理行動全体の評価指標を構築できれば,より総合的な調理技能の分析が可能となると考えられる.
このような応用範囲の拡大は,家庭料理だけでなく,プロの調理現場や教育機関における技能習得での活用も考えられる.

最後に,誤判定への対応である.
現在のシステムでは,包丁を持つ動作や準備動作が切る動作として誤判定されるケースが存在する.
この問題に対処するためには,センサーデータの前処理を含む工夫が必要である.
例えば,センシングの開始前に包丁をあらかじめ持つ運用ルールを設けるなどの対策で誤判定のリスクを軽減できる.
しかし,この対策を実施する際には,安全性への十分な配慮が必要である.
特に,包丁を持った状態で行える安全なジェスチャーを検討し,切り始める前と後で確実に区別できる仕組みを設計する必要がある.
また,センサーの感度や位置の調整,さらには特徴量の選定やアルゴリズムの改良を進めるなど,誤判定を最小限に抑える手法の考案も重要な課題となる.

さらに今後の展望として,特徴量を抽出する際に使用した3次元空間での可視化システムを,結果のフィードバックへの応用としての使用を検討している.
具体的には,抽出した特徴量だけでなく,調理動作そのものを可視化し,他者との比較を行う手法を模索する.
この可視化によって,特徴量の変化だけを提示する場合と,動作そのものを提示する場合で,被験者の意識や行動への影響にどのような違いが生じるかの調査が可能になると考えられる.

以上の課題に取り組むと,調理行動解析の精度が向上し,応用範囲の拡大が期待される.