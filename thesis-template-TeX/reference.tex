\thispagestyle{myheadings}
\addcontentsline{toc}{chapter}{\protect\numberline {} 参考文献の書き方例}
    
\begin{thebibliography}{参考文献}

	\bibitem{携帯電話搭載センサによるリアル
	タイム生活行動認識システム}
	大内 一成, 土井 美和子:携帯電話搭載センサによるリアル
	タイム生活行動認識システム, 情報処理学会論文誌, Vol. 53,
	No. 7, pp. 1675-1686 (2012).
	
	\bibitem{ウェアラブルセンサを用いた熟練指導員のヤスリがけ技能主観評価値の再現}
	榎堀 優, 間瀬 健二:ウェアラブルセンサを用いた熟練指導員のヤスリがけ
	技能主観評価値の再現, 人工知能学会論文誌, Vol. 28, No. 4, pp. 391-399 (2013).
	
	\bibitem{実世界に広がる装着型センサを用いた行動センシングとその応用:6. 装着型センサを用いた運転者行動センシング}
	多田 昌裕:実世界に広がる装着型センサを用いた行動センシングとその応用:6. 装着型センサを用いた運転者行動センシング, 情報処理, Vol. 54, No. 6, pp. 588-591 (2013).
	
	\bibitem{加速度センサを用いた包丁技術向上支援システムの提案}
	小林 花菜乃, 加藤 岳大, 横窪 安奈, ロペズ ギヨーム:
	加速度センサを用いた包丁技術向上支援システムの提案, マルチメディア,分散協調とモバイルシンポジウム論文集(DICOMO2020), Vol. 2020, pp. 1000-1003 (2020).
	
	\bibitem{マ
	ルチモーダルセンシングに基づく料理中のマイクロ行動認識
	の提案}
	石山 時宗, 松井 智一, 藤本 まなと, 諏訪 博彦, 安本 慶一:マ
	ルチモーダルセンシングに基づく料理中のマイクロ行動認識
	の提案, 情報処理学会関西支部支部大会講演論文集, Vol. 2021, (2021).
	
	\bibitem{手首装着型の加速度センサを用いた実時間調理行動認識手法の実現}
	大神 健司, 飛田 博章:手首装着型の加速度センサを用いた実時間調理行動認識手法の実現,
	 人工知能学会全国大会論文集, Vol. 37, (2023).
	
	\bibitem{kumazawaanalysis}
	Ayato Kumazawa, Fuma Kato, Katsuhiko Kaji, Nobuhide Takashima, Katsuhiro Naito, Naoya Chujo, and Tadanori Mizuno:
	Analysis and Sharing of Cooking Actions Using Wearable Sensors, 
	\textit{International Workshop on Informatics}, Vol. 17, (2023).
\end{thebibliography}
% Local Variables: 
% mode: japanese-LaTeX
% TeX-master: "root"
% End: 
