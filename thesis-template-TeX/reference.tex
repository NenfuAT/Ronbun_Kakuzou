\thispagestyle{myheadings}
\addcontentsline{toc}{chapter}{\protect\numberline {} 参考文献の書き方例}
    
\begin{thebibliography}{参考文献}

	\bibitem{マイクロブログにおけるユーザの属性と習慣行動の推定に関する研究}
	加藤 諒, 中村 健二, 山本 雄平, 田中 成典, 坂本 一磨他:マイクロブログにおけるユーザの属性と習慣行動の推定に関する研究, 情報処理学会論文誌, Vol. 57, No. 5, pp. 1421–1435 (2016).

	\bibitem{加速度センサを活用した非装着型の人間の行動推定システム}
	内田 泰広, 澤本 潤, 杉野 栄二:加速度センサを活用した非装着型の人間の行動推定システム, 電子情報通信学会技術研究報告; 信学技報, Vol. 115, No. 232, pp. 1–6 (2015).

	\bibitem{単一3軸加速度センサによる行動推定}
	赤堀 顕光, 岸本 圭史, 小栗 宏次:単一3軸加速度センサによる行動推定, 電子情報通信学会技術研究報告; 信学技報, Vol. 105, No. 456, pp. 49–52 (2005).

	\bibitem{加速度センサから収集した人間行動データのサンプリング周波数推定手法}
	河野 日生, 岡本 真梨菜, 村尾 和哉:加速度センサから収集した人間行動データのサンプリング周波数推定手法, マルチメディア,分散,協調とモバイルシンポジウム 2023 論文集, Vol. 2023, pp. 589–596 (2023).

	\bibitem{小型センサモジュール搭載シューズを用いた行動センシング}
	島内 岳明, 勝木 隆史, 豊田 治:小型センサモジュール搭載シューズを用いた行動センシング, マイクロエレクトロニクスシンポジウム論文集 第28回マイクロエレクトロニクスシンポジウム, pp. 307–310 一般社団法人エレクトロニクス実装学会 (2018).

	\bibitem{携帯電話搭載センサによるリアル
	タイム生活行動認識システム}
	大内 一成, 土井 美和子:携帯電話搭載センサによるリアル
	タイム生活行動認識システム, 情報処理学会論文誌, Vol. 53,
	No. 7, pp. 1675-1686 (2012).

	\bibitem{気圧センシング技術を用いた行動認識手法}
	米田 圭佑, 望月 祐洋, 西尾 信彦:気圧センシング技術を用いた行動認識手法, 情報処理学会論文誌, Vol. 56, No. 1, pp. 260–272 (2015).

	\bibitem{WiFiバックスキャッタータグを用いた非接触生活行動認識システムの提案}
	伊勢田 氷琴, 安本 慶一, 内山 彰, 東野 輝夫:WiFiバックスキャッタータグを用いた非接触生活行動認識システムの提案, マルチメディア,分散,協調とモバイルシンポジウム 2023 論文集, Vol. 2023, pp. 1193–1203 (2023).

	\bibitem{Multimodal Wearable Sensing for Fine-Grained Activity Recognition in Healthcare}
	De, D., Bharti, P., Das, S. K. and Chellappan, S.: Multimodal Wearable Sensing for Fine-Grained Activity Recognition in Healthcare, IEEE Internet Computing, Vol. 19, No. 5, pp. 26–35 (2015).

	\bibitem{GLPose: Global-Local Representation Learning for Human Pose Estimation}
	Jiao, Y., Chen, H., Feng, R., Chen, H., Wu, S., Yin, Y. and Liu, Z.: GLPose: Global-Local Representation Learning for Human Pose Estimation, ACM Trans. Multimedia Comput. Commun. Appl., Vol. 18, No. 2s (2022).

	\bibitem{3次元点群を用いたマイクロ行動認識手法の提案}
	三嶋 祐輝, 松井 智一, 松田 裕貴, 諏訪 博彦, 安本 慶一:3次元点群を用いたマイクロ行動認識手法の提案, Technical Report 37 (2023).

	\bibitem{照度センサを用いた在宅時の活動見守りシステム}
	松原 裕之:照度センサを用いた在宅時の活動見守りシステム, 電気学会論文誌C(電子・情報・シス テム部門誌), Vol. 143, No. 8, pp. 754–759 (2023).

	\bibitem{Human Action Recognition Method Based on Wearable Inertial Sensor}
	iang, Z.: Human Action Recognition Method Based on Wearable Inertial Sensor, in Proceedings of the 2021 6th International Conference on Cloud Computing and Internet of Things, CCIOT ’21, pp. 73–76, New York, NY, USA (2021), Association for Computing Machinery.

	\bibitem{Real-Time Joint Axes Estimation of the Hip and Knee Joint during Gait Using Inertial Sensors}
	Norden, M., Műller, P. and Schauer, T.: Real-Time Joint Axes Estimation of the Hip and Knee Joint during Gait Using Inertial Sensors, in Proceedings of the 5th International Workshop on Sensor-Based Activity Recognition and Interaction, iWOAR ’18, New York, NY, USA (2018), Association for Computing Machinery.

	\bibitem{歩行者自律測位における行動センシング知識の利用}
	村田 雄哉, 梶 克彦, 廣井 慧, 河口 信夫:歩行者自律測位における行動センシング知識の利用, マルチメディア, 分散協調とモバイルシンポジウム2014論文集, Vol. 2014, pp. 1614–1619 (2014).
	
	\bibitem{ウェアラブルセンサを用いた熟練指導員のヤスリがけ技能主観評価値の再現}
	榎堀 優, 間瀬 健二:ウェアラブルセンサを用いた熟練指導員のヤスリがけ
	技能主観評価値の再現, 人工知能学会論文誌, Vol. 28, No. 4, pp. 391-399 (2013).
	
	\bibitem{実世界に広がる装着型センサを用いた行動センシングとその応用:6. 装着型センサを用いた運転者行動センシング}
	多田 昌裕:実世界に広がる装着型センサを用いた行動センシングとその応用:6. 装着型センサを用いた運転者行動センシング, 情報処理, Vol. 54, No. 6, pp. 588-591 (2013).

	\bibitem{装着型加速度センサを用いた運転中の行動推定}
	茅嶋 伸一郎, 秋月 拓磨, 荒川 俊也, 高橋 弘毅:装着型加速度センサを用いた運転中の行動推定, 知能と情報, Vol. 34, No. 2, pp. 544–549 (2022).


	% こっから料理
	\bibitem{加速度センサを用いた包丁技術向上支援システムの提案}
	小林 花菜乃, 加藤 岳大, 横窪 安奈, ロペズ ギヨーム:
	加速度センサを用いた包丁技術向上支援システムの提案, マルチメディア,分散協調とモバイルシンポジウム論文集(DICOMO2020), Vol. 2020, pp. 1000-1003 (2020).
	
	\bibitem{マ
	ルチモーダルセンシングに基づく料理中のマイクロ行動認識
	の提案}
	石山 時宗, 松井 智一, 藤本 まなと, 諏訪 博彦, 安本 慶一:マ
	ルチモーダルセンシングに基づく料理中のマイクロ行動認識
	の提案, 情報処理学会関西支部支部大会講演論文集, Vol. 2021, (2021).

	\bibitem{Cooking Activities Recognition in Egocentric Videos Using Hand Shape Feature with Openpose}
	Okumura, T., Urabe, S., Inoue, K. and Yoshioka, M.: Cooking Activities Recognition in Egocentric Videos Using Hand Shape Feature with Openpose, in Proceedings of the Joint Workshop on Multimedia for Cooking and Eating Activities and Multimedia Assisted Dietary Management, CEA/MADiMa ’18, pp. 42–45, New York, NY, USA (2018), Association for Computing Machinery.

	\bibitem{NOSE: A Novel Odor Sensing Engine for Ambient Monitoring of the Frying Cooking Method in Kitchen Environments}
	Khaloo, P., Oubre, B., Yang, J., Rahman, T. and Lee, S. I.: NOSE: A Novel Odor Sensing Engine for Ambient Monitoring of the Frying Cooking Method in Kitchen Environments, Proceedings of the ACM on Interactive, Mobile, Wearable and Ubiquitous Technologies, Vol. 3, No. 2, pp. 1–25 (2019).

	\bibitem{A dataset for complex activity recognition with micro and macro activities in a cooking scenario}
	Lago, P., Takeda, S., Alia, S. S., Adachi, K., Benaissa, B., Charpillet, F. and Inoue, S.: A dataset for complex activity recognition with micro and macro activities in a cooking scenario, preprint (2020).

	\bibitem{手首装着型の加速度センサを用いた実時間調理行動認識手法の実現}
	大神 健司, 飛田 博章:手首装着型の加速度センサを用いた実時間調理行動認識手法の実現, 人工知能学会全国大会論文集, Vol. 37, (2023).

	\bibitem{kumazawaanalysis}
	Ayato Kumazawa, Fuma Kato, Katsuhiko Kaji, Nobuhide Takashima, Katsuhiro Naito, Naoya Chujo, and Tadanori Mizuno:
	Analysis and Sharing of Cooking Actions Using Wearable Sensors, 
	\textit{International Workshop on Informatics}, Vol. 17, (2023).

\end{thebibliography}
% Local Variables: 
% mode: japanese-LaTeX
% TeX-master: "root"
% End: 
