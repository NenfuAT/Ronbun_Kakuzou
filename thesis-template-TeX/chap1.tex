\chapter{はじめに}
\section{背景}
近年ではスマートウォッチと呼ばれる腕時計型のデバイスやセンサグローブと呼ばれる手袋型のデバイスなど,動作を取得できるウェアラブルデバイスが普及し始めている.
これらのデバイスは,小型化と高性能化が進み,一般消費者にも手軽に利用可能となりつつあり,装着者の腕や手の動きを容易かつ詳細に取得できる.
総務省の調査3ではスマートフォンの世帯保有率は年々増加しており,2022 年では 90.1\%となっている.
MMD研究所のスマートウォッチの所持率調査では2019年4は18.0\%(n=1867)に対して2021年5は38.0\%(n=658)とスマートウォッチも普及してきているのがわかる.
またスマートウォッチについては内蔵されているセンサの数も増加してきている.
現行のスマートウォッチであるPixelWatch2の場合GPS・コンパス・高度計・酸素飽和度計測用赤色および赤外線センサ・多目的電気センサ・マルチパス光学式心拍数センサ・3軸加速度計・ジャイロスコープ・周囲光センサ・皮膚コンダクタンスを測定する電気センサ・皮膚温センサ・気圧計・磁力計が内蔵されておりスマートフォンに引けを取らないセンサの量がある.

取得されたデータは,単なる動作記録にとどまらず,さまざまな用途で活用されている.例えば,スポーツ分野ではアスリートの動きを詳細に計測し,そのデータをプロの動作と比較してフォーム改善によるパフォーマンス向上や,怪我のリスク低減を目指している.
さらに,医療分野では,患者のリハビリテーション支援や日常生活動作の監視にウェアラブルデバイスが利用されており,歩行解析や関節可動域の測定を通じて回復状況を定量的に評価している.
これにより,従来の主観的な評価に比べて,より客観的で精密な技能評価が実現しており,リハビリプランの最適化や治療効果の測定に役立てられている.

このように,スポーツや医療分野では専門的な動作の技能評価が進んでいる一方で,日常生活に関連する動作の評価については,まだ十分な研究が進んでいない.
日常動作の中でも技能が必要とされるものとして,調理動作が挙げられる.調理は単なる日常作業ではなく,熟練度に応じた差が明確に現れる技能であり,その評価や分析の対象として注目されている.
調理技能の評価では,ウェアラブルデバイスを活用して調理動作を詳細に記録し,分析を行える.
これにより,上達の過程や改善点を客観的に把握できるようになり,自己評価やモチベーションの向上が期待される.
例えば,現在の自分の動作を過去の記録と比較する際,調理技術が向上している様子を視覚的に確認でき,さらなる向上心を引き出せる.
また,熟練者との動作比較では,自身の動作における課題が明確になり,より効率的に技能向上を目指せると考えられる.

\section{目的とアプローチ}
本研究では,包丁を持った手の3次元的な動きをセンシングし,取得したデータを基にした特徴量抽出を目的とする.
本研究による提案手法は図\ref{fig:1}に示す.
\figimage{images/fig1.pdf}{150}{本研究の概要図}{fig:1}
アプローチとして,ウェアラブルセンサで取得した加速度データおよび角速度データを活用し,使用者の腕の動きを3次元的に推定する.
推定された動作データを詳細に分析し,切った回数や切るペースといった定量的な特徴量を抽出する.
この特徴量は,過去の自分との比較を通じた自己改善だけでなく,熟練者との比較による具体的な改善ポイントの把握にも役立つ.
これにより,使用者が調理技能を効果的に向上させるための支援ツールとしての活用が期待される.