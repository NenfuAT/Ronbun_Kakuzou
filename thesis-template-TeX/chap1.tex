\chapter{はじめに}
\section{背景}
近年ではスマートウォッチと呼ばれる腕時計型のデバイスやセンサグローブと呼ばれる手袋型のデバイスなど,動作を取得できるウェアラブルデバイスが普及し始めている.
これらのデバイスは,小型化と高性能化が進み,一般消費者にも手軽に利用可能となりつつあり,装着者の腕や手の動きを容易かつ詳細に取得できる.
総務省の調査\footnote[1]{ https://www.soumu.go.jp/johotsusintokei/whitepaper/ja/r05/html/nd24b110.html}ではスマートフォンの世帯保有率は年々増加しており,2022 年では 90.1\%となっている.
MMD研究所のスマートウォッチの所持率調査では2019年\footnote[2]{https://mmdlabo.jp/investigation/detail 1812.html}は18.0\%(n=1867)に対して2021年\footnote[3]{https://mmdlabo.jp/investigation/detail 1930.html}は38.0\%(n=658)とスマートウォッチも普及してきているのがわかる.
またスマートウォッチについては内蔵されているセンサの数も増加してきている.
現行のスマートウォッチであるPixelWatch2の場合GPS・コンパス・高度計・酸素飽和度計測用赤色および赤外線センサ・多目的電気センサ・マルチパス光学式心拍数センサ・3軸加速度計・ジャイロスコープ・周囲光センサ・皮膚コンダクタンスを測定する電気センサ・皮膚温センサ・気圧計・磁力計が内蔵されておりスマートフォンに引けを取らないセンサの量がある.

取得されたデータは,単なる動作記録にとどまらず,さまざまな用途で有効に活用されており,その利用範囲は急速に拡大している.
その一例として,スポーツ分野では,アスリートの動きが詳細に計測され,そのデータをプロの選手の動作との比較により,フォーム改善を通じたパフォーマンスの向上を図るとともに,怪我のリスク低減を目指している.
こうした計測データをもとに,選手は自分の動作を客観的に把握でき,トレーニングの際にどの部分が改善されるべきかを特定できる.
また,動作を最適化するために,トレーニング方法や休養の取り方を個別に調整でき,その結果として競技パフォーマンスの向上や怪我の予防が実現される.
さらに,ウェアラブルデバイスを活用したリアルタイムのフィードバックが可能となるため,トレーニング中に即座に修正すべきポイントを認識し,効率的に技術向上を図っている.
また,医療分野においては,患者のリハビリテーション支援や日常生活動作の監視にウェアラブルデバイスが広く利用されている.
歩行解析や関節可動域の測定,さらには心拍数や体温などの生体データも同時に取得され,患者の回復状況を定量的に評価している.
これにより,患者の状態をリアルタイムで把握が可能となり,治療方針を迅速に定められる.
例えば,患者が日常生活で行う動作の中で問題点がある場合,それを早期に発見でき,適切なリハビリテーションや治療方法の提案が可能となる.
さらに,こうしたデータは医師やリハビリスタッフにとっても貴重な指標となり,患者一人ひとりに合わせた最適なケアを提供するための重要な情報源となっている.
これらの技術の導入により,従来の主観的な評価方法に比べて,より客観的で精密な技能評価が可能となった.
従来の評価方法では,評価者の経験や感覚に頼る部分が大きく,個人差の発生が避けられなかったが,ウェアラブルデバイスを用いた評価では,センサーデータに基づく客観的な判断が可能となり,より正確な評価が実現する.
この客観的な評価を基に,リハビリプランの最適化が進み,患者の回復状況に応じて個別に調整された治療が提供できるようになっている.
また,これにより治療効果の測定も精度高く行えるため,患者にとって最適な治療計画を立てるための基盤が整う.
結果として,医療従事者はより効果的かつ効率的に治療ができ,患者一人ひとりの個別ニーズに応じた対応が可能となっている.
これらの技術が進化すると,リハビリテーションの過程やスポーツトレーニングにおいて,より個別的で効果的なアプローチが実現し,従来の手法に比べて遥かに高い成果を上げられるようになっている.

このように,スポーツや医療分野では専門的な動作の技能評価が進んでいる一方で,日常生活に関連する動作の評価については,まだ十分な研究が進んでいない.
日常動作の中でも技能が必要とされるものとして,調理動作が挙げられる.調理は単なる日常作業ではなく,熟練度に応じた差が明確に現れる技能であり,その評価や分析の対象として注目されている.
調理技能の評価では,ウェアラブルデバイスを活用して調理動作を詳細に記録し,分析を行える.
これにより,上達の過程や改善点を客観的に把握できるようになり,自己評価やモチベーションの向上が期待される.
例えば,現在の自分の動作を過去の記録と比較する際,調理技術が向上している様子を視覚的に確認でき,さらなる向上心を引き出せる.
また,熟練者との動作比較では,自身の動作における課題が明確になり,より効率的に技能向上を目指せると考えられる.

\section{目的とアプローチ}
本研究では,包丁を持った手の3次元的な動きをセンシングし,取得したデータを基に特徴量の抽出を目的とする.
本研究で提案する手法の概要は図\ref{fig:1}に示されている.
\figimage{images/fig1.pdf}{150}{本研究の概要図}{fig:1} 
本アプローチでは,ウェアラブルセンサを用いて取得した加速度データおよび角速度データを活用し,使用者の腕の動きを3次元空間で推定する.
センサによって収集されたデータは,腕の各部分の動きに関する詳細な情報を提供し,これらのデータを基に高精度な動作推定が可能となる.

この動作データをさらに詳細に分析し,例えば,切った回数や切るペース,包丁の動きの滑らかさ,動作の一貫性といった定量的な特徴量を抽出する.
これらの特徴量は,使用者の動作がどれほど効率的であるか,またはどの部分に改善の余地があるかを明確に示す指標となる.
特徴量の抽出により,単なる動作の記録を超えたフィードバックを得られ,技能向上のための重要な手掛かりを提供する.

これらの特徴量は,過去の自分のデータと比較を行い自己改善を促進するだけでなく,熟練者のデータと比較も行い,具体的な改善ポイントや最適な動作パターンを把握する手助けにもなる.
こうした比較により,使用者は自分の動作のどこに不足があり,どの部分を改善すべきかを具体的に理解でき,技術的な向上に向けた効率的なアプローチが可能となる.

最終的には,これらのデータをリアルタイムで処理し,使用者に対してフィードバックを提供できれば,調理技能を効果的に向上させるための支援ツールとしての活用が期待される.
これにより,使用者は目標に向かって着実に練習を積み重ね,より高いレベルでの調理技術の習得が期待できる.

\section{論文構成}
本論の構成を以下に示す.
2章では関連研究を示し3章では本研究の全体像の説明と特徴量抽出手法について述べる.
4章では提案した特徴量抽出手法を用いて作成した推定システムについて示し,5章で推定システムを使用した評価実験について述べる.
最後に6章として本論文のまとめと本研究の今後の課題について述べる.