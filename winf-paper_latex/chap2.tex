\section{関連研究}
ウェアラブルデバイスを用いた技能評価や技能向上に関する研究がある.
スマートフォン内蔵のセンサを使用して生活行動を推定する研究
\cite{携帯電話搭載センサによるリアルタイム生活行動認識システム}
ウェアラブルセンサを用いたヤスリがけ動作の技能評価による熟達者の動作の再現に関する研究,
\cite{ウェアラブルセンサを用いた熟練指導員のヤスリがけ技能主観評価値の再現}
,ウェアラブルセンサを用いた運転技能評価に関する研究
\cite{実世界に広がる装着型センサを用いた行動センシングとその応用:6. 装着型センサを用いた運転者行動センシング}
などがある.
これらの研究ではウェアラブルデバイスを用いて集めたデータをもとに技能評価を行い使用者の技術向上を目指している.
本研究では調理動作を対象する技能評価のための特徴量を抽出する.

調理行動の推定に専用の機材を用いる研究がある.
包丁に直接加速度センサを取り付け包丁技術を判定する研究\cite{加速度センサを用いた包丁技術向上支援システムの提案},マルチモーダルセンシングによる料理中
のマイクロ行動の認識を目指す研究\cite{マルチモーダルセンシングに基づく料理中のマイクロ行動認識の提案}
,加速度センサを腕に取り付けて調理動作の判定を行う研究\cite{手首装着型の加速度センサを用いた実時間調理行動認識手法の実現}がある.
これらの研究は主に調理機材やキッチン,人間にセンサ類を取り付けるなど,自作の装置を用いて調理動作を推定している.
また,ウェアラブルデバイスを用いた調理の切る動作の分析の研究がある\cite{kumazawaanalysis}.
しかし,この研究では分析の際に加速度しか使用していない.
本研究ではウェアラブルデバイスに内蔵されている加速度・角速度センサを用いて調理動作の推定を行う.
ウェアラブルデバイスを用いるとセンサ等を準備する必要がなくなり専門的な機材がなくてもデータの収集が可能である.

