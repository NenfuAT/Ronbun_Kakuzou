\section{はじめに}
%(仮)
料理のレシピを共有する方法にはさまざまな種類があり世の中
に浸透している.インターネットが普及し情報の共有が簡単になっ
た今ではインターネットでレシピが見られるようになっている.共
有方法として自分のブログにレシピを載せたり,レシピ共有サー
ビスの利用などがある.レシピ共有サービスとしてクックパッド\footnote{URL}
やクラシル\footnote{URL}などがある.また,レシピを利用して作成したものを
報告する機能がある.クックパッドでは「つくれぽ」,クラシル
では「たべれぽ」が該当する.
レシピを共有する際に調理作業中のデータも共有し,他の人と
比較すると今後の調理に対するモチベーションの増加を狙えると
考えた.他の人はレシピを共有した作者や自分自身などが考えら
れる.作業データを共有すると自分はレシピ作者と比べてどのく
らい調理作業が違うのか,過去の自分よりどのくらいうまくなっ
たのかなどの比較ができる.例えば今と過去の自分を比較した場
合調理が上達しているのがわかりやすくなり,もっと上手くなろ
うとするモチベーションが増加すると考えられる.またプロと比
較した場合も改善点が見つかり調理の上達に近づけるのではない
かと考えられる.

先行研究ではアプローチとしてウェアラブル
センサを使い調理作業をセンシングを行い比較用の特徴量を抽出
していたが,加速度センサのみを使用しており簡単な調理動作にしか対応していなかった.
そこで本研究では複数のセンサを用いたより高度な調理動作に対応したシステムの作成を目的とする.