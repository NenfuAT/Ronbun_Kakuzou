\section{はじめに}

情報学ワークショップは,東海地区を中心として大学・企業等の情報技術に関する最新の研究成果を発表すると共に,学生・研究者の交流の場を提供することを目的としております.本年度は研究論文発表及びポスター発表で行います.研究論文の応募に際しては,2段組4ページ以内とし,ポスター発表に関しては300字の抄録のみでの申込となります.研究論文に提出いただいた論文は,例年通り審査委員により評価します.さらに提出論文の中から特に優秀な論文を選考しこれを表彰します.情報学分野の情報交換の場として多くの方のご参加を期待しております.

本資料では情報学ワークショップの研究論文の作成方法を示します.
発表者の方は,この資料に準拠して研究論文の原稿を作成していただくようお願いいたします.
ただし,それぞれの専門分野で優先すべきフォーマットなどがありましたら,そちらのフォーマットに従うマイナーチェンジも結構です.
