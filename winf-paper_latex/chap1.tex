\section{はじめに}
近年ではスマートウォッチと
呼ばれる腕時計型のデバイスやセンサグローブと呼ばれる手袋型のデバイスなど動作を取得できるウェアラブルデバイスが普及し始めている.
それに伴い装着者の腕の動きを容易に取得可能となった.
取得したデータはさまざまな用途で使用されている.
例として,スポーツではアスリートの動きを詳細に計測してプロの動作と比較を行いパフォーマンスの向上や怪我の予防に役立てられている.
また,医療分野では,患者のリハビリテーションや日常生活動作を監視するためにウェアラブルデバイスが使用されている.
歩行解析や関節の可動域の測定を行い,回復状況を定量的に評価が可能となっている.
従来の主観的な評価に比べて,より客観的で精密な技能評価が可能になっており,リハビリプランの最適化や治療の効果測定に役立てられている.

スポーツや医療のように専門的な動作の技能評価は進んできているが,日常生活に関係してくる動作を評価しているものはまだ少ない.
日常動作の中で技能が必要となってくる動作の一つに調理があげられる.
調理技能の評価においても,これらのデバイスを用いて調理動作を詳細に記録・分析を行い,上達の過程や改善点を客観的に捉えられる.
例えば今と過去の自分を比較した場合調理技術が向上しているのがわかりやすくなり,向上心が増加すると考えられる.また熟練者と調理動作を比
較した場合も改善点が見つかり調理技能の向上に近づけると考えられる.

そこで本研究では包丁を持った手の3次元的な動きをセンシングしたデータを用いて算出し,特徴量の抽出を目的とする.
抽出した特徴量は過去の自分との比較や熟練者との比較により動作の改善を促進し,使用者の調理技能向上に使用する.
アプローチとしてウェアラブルセンサを用いて取得した加速度・角速度から使用者の腕の動きを3次元的に推定する.
推定した3次元の動きを分析し切った回数や切るペースなどを抽出する.