\section{加速度と角速度センサを用いた刺身の平造り推定}
本章では先行研究の問題点と,本研究での目的について述べる.
\subsection{先行研究の課題}
先行研究として{ここに綾人さんのやつ}があるが加速度のみのシステムである.
様々な調理動作が存在しているが加速度のみで推定可能な動作は少なく,きゅうりの輪切りなどの簡単な動作しか対応していない.
そこで本研究では複数のセンサを用いて端末の動きを推定し特徴量として扱うシステムの作成を目指す.
\subsection{要件定義}
今までのシステムが対応していない動作として刺身の平造りがあげられる.
推定できない理由としてはきゅうりの輪切りのような包丁とまな板が音を上げてぶつかる動作とは異なり,
刺身の平造りは奥から手前に引くように切るため包丁がまな板と衝突した際に発生する加速度を特徴量として扱えなかった.

また,刺身を切る際の評価基準として次のようなものが使用できると考える.
例えば,下手な人は切る際に包丁を小刻みに動かしてしまい身が崩れてしまい食感が悪くなってしまう.
一方上手い人は一回の包丁を引く動作で刺身を切り出しており断面が滑らかで美味しくなると言われている.
そこで端末の移動方向や距離が推定できれば切り込んだ回数を求めることができ比較ができると考え,
加速度と角速度から端末の3次元的な移動を推定するシステムを作成した.


