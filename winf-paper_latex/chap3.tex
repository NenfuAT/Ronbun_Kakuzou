\section{論文本体のフォーマット概要}

基本的にこのフォーマットに準拠していただきますが,もちろん書きやすいように,読みやすいようにマイナーチェンジして頂いてもOKです.
ただし最終的に書式を実行委員会・プログラム委員会で統一させて頂く場合があります.
なお,この \LaTeX 版のテンプレートは,MS-Word版と異なる点もあります.
以下の各項目を目安として考えていただければと思います.

\begin{enumerate}
\item 日本語タイトル,著者名,所属,最初に1段組みで書きます.
文字サイズは,日本語タイトル12ポイント,日本語の著者名と所属10.5ポイントです.
この間の行間は15ポイントの固定値になっています.

\item 本文は2段組みで,フォントは8ポイントで行間は1行です.
この結果,1ページがおおよそ28文字×57行x2段となります.

\item 各章の見出しは10ポイント,各節の見出しは9ポイントとしています.

\item マージンは上下が3cm,左右が2cmとします.

\item MS-Word版では,文字フォントは,タイトル,著者名,所属,見出しの部分が,MSゴシックとArialを使用しています.
また本文は,MS明朝とCenturyです.
ただしフォントについては特に制限はいたしませんので,独自の形式で論文を作成して構いません.

\item \underline{ページ番号をつけないでください.}

\item この資料は,\LaTeX の範囲内で,おおむねフォーマットに従っているつもりです.
\end{enumerate}