\section{おわりに}
本章では本論のまとめと今後の課題について述べる.
\subsection{まとめ}
本研究では調理動作のデータを分析し使用者の技能評価を行い,過去の自分との比較や熟練者との比較により調理技能向上を目的としている.
アプローチとしてウェアラブルデバイスを用いて調理作業のデータから腕の動きを算出し特徴量の抽出に利用している.
研究の全体像として「調理動作の収集」「データの分析」「可視化によるフィードバック」の項目に分けられる.
スマートウォッチを使用し取得した加速度・角速度をもとに腕の動きを算出している.
算出した腕の移動・角度を使用して調理動作における特徴量を抽出している.
抽出した特徴量の可視化を行い調理技能の向上を促す.
基礎検討として切る工程における特徴量の抽出を行った.
その結果得られた特徴量をもとに技能の評価や比較が可能と考えられる.
\subsection{今後の課題}
今後の課題として抽出を行った特徴量をもとにした他者との比較がある.
今回の調査では特徴量を抽出したのみで他者との比較から調理技能の向上につながるかは検証できていない.
検証のためには他者との比較から調理に対する意識の変化など調理技能の向上と関係する変化が見られるか実験を行う必要がある.

特徴量の抽出については切る動作のみしか行っていないため他の調理動作でも特徴量を算出できるかについて調査する必要がある.
炒める工程・混ぜる工程などがあるのでそれぞれについて調査する.


切る動作においても包丁を持つ際などの動きが切っていると誤判定される問題が残っているため,センシング開始前に包丁をあらかじめ持っておくなどの対策を行う必要がある.
その際安全性にも考慮する必要があるため切り始める前と後に包丁を持った状態でも行える安全なジェスチャーを考える必要がある.

また特徴量を抽出する際に使用した3次元空間での可視化システムについては結果のフィードバックに使用できる可能性がある.
抽出した特徴量のみでなく調理動作そのものを可視化して他者との比較を行う.
これにより特徴量を可視化するだけの場合と意識の変化に違いが見られるか検討できる.