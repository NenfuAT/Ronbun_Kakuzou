\section{刺身の平造り動作の分析}
調理動作の分析について刺身を切り分ける工程を例にして,どのような特徴が見られるかを調べた.
\subsection{調査内容}
加速度と角速度を収集する自作アプリを起動したスマートウォッチ(PixelWatch2)を包丁を握る側の腕に装着しセンシングを行う.
被験者は10cmほどの魚の切り身を刺身に切り分ける調理を行う.
刺身を切る動作は刃を入れる角度による断面の変化や包丁を引く回数から切り分けた身の分厚さなどの複数の評価基準が得られる.
また,データのラベリングのために動画を撮影しながらデータ収集を行った.
\subsection{データの分析}

結果として得られた包丁の動作の一例を表したグラフを次の図\ref{fig:4}に示す.
\figimage{images/fig4.pdf}{80}{切るの動作のグラフ}{fig:4}
このデータは3章で述べた推定手法を用いて求めた端末の回転量から変換した角度と移動を表している.
誤差の軽減のためにハイパスフィルタを使用した.
赤くラベリングされている部分が実際に刺身を切り分けている動作である.
また,青くラベリングされている部分は包丁を持つ動作と置く動作である.

3次元的な動きを扱うにあたりグラフで可視化してもわかりづらい.
そこで求めた回転と移動を3次元空間での可視化を行う図\ref{fig:5}のようなシステムを作成した
\figimage{images/fig5.pdf}{80}{3次元空間での可視化システム}{fig:5}
この可視化システムによって,現実の動きとの比較が容易になり特徴量の抽出を行いやすくなった.

調理工程のラベルは撮影した動画から手入力している.
動画との同期は3次元空間での可視化システムで見比べて行っている.
特に変化が見られたY座標軸の移動距離と推定した切ったタイミングを表したグラフが次の図\ref{fig:6}である.

\figimage{images/fig6.pdf}{80}{推定結果のグラフ}{fig:6}

まずは切った回数の推定を行う.
動画によるラベリングとデータにより,切る際には一度奥(+)方向へ移動してから手前(-)方向へ移動しているとわかる.
そのため今回は極値検出を用いて切った回数の推定を行う.
切る際にメジャーを用い手首の移動距離を調べ,一番動かしていなかったところを基準に閾値を2cmと定めた.
閾値以上となった極大値から次の極大値までの区間内で一番値が小さな極小値までで1回切ったと推定する.
その結果を図\ref{fig:6}で「切り始め」と「切り終わり」として表している.
結果から切ったと推定した部分とラベリングを行なった区間が一致したことがわかる.
以上より切る動作を推定できたと考えられる.
しかし包丁を持つ動作と置く動作も切っている部分と判定されてしまう場合があるため,あらかじめ包丁を持ってからセンシングを始めるなど改善する必要がある.

次に平均ペースを求める.
切るペースは1回1回の間の秒数を利用する.
また,その際のRoll(x)軸での平均角度も特徴量として使用する.
今回のデータの特徴量を包丁を持つ動作と置く動作を除きを表にしたものを次の表\ref{table:1}に示す.
\begin{table}[ht]
    \centering
    \caption{特徴量}
    \label{table:1}
    \resizebox{70mm}{!}{ 
        \begin{tabular}{ccc}
            \hline\hline
            切った回数 & かかった時間 & 角度の平均 \\
            \hline
            1 & C1 & R1 \\
            2 & C2 & R2 \\
            3 & C1 & R1 \\
            4 & C2 & R2 \\
            5 & C1 & R1 \\
            \hline
            全体の平均 & C1 & R1 \\
            \hline
        \end{tabular}
    }
\end{table}
表\ref{table:1}の結果から得られた特徴量は次のような評価の基準になると考えられる.
まず切った回数から刺身の厚さ,平均ペースから手際の良さ,切っている際の包丁の角度から断面の大きさなどを算出できると考えられる.
また,今回は参照していないが距離に変換する前の線形加速度を参照し一回の切り込み時の包丁のブレ等も算出可能と考えられる.

