\section{刺身の平造り動作の分析}
分析について切る工程を例にして,実際にどのような特徴が見られるかを調べた.
\subsection{調査内容}
センシング方法としてスマートウォッチ(PixelWatch2)
を包丁を握る側の腕に装着し自作したアプリで加速度と角速度を収集した.
被験者には10cmほどの魚の切り身を刺身にしてもらった.
刺身を切る動作を分析の対象にした理由は,刃を入れる角度による身の厚みの変化や
包丁を引く回数で断面の滑らかさなどの評価基準が考えられたからである.
また,データのラベリングのために動画を撮影しながらデータ収集を行った.
\subsection{データの分析}

結果として得られた包丁の動作の一例を表したグラフを次の図に示す.
\figimage{images/fig3.pdf}{70}{切るの動作のグラフ}{fig:3}
このデータは3章で述べた推定手法を用いて求めた端末の回転量から変換した角度と移動を表している.
赤くラベリングされている部分が実際に刺身を切り分けている部分である.
また,青くラベリングされている部分は包丁を持つ動作と置く動作である.

3次元的な動きを扱うにあたり扱うグラフで可視化してもわかりずらい.
そこで図\ref{fig:4}のような求めた回転と移動を3D空間での可視化を行うシステムを作成した.
\figimage{images/fig4.pdf}{70}{3D空間での可視化システム}{fig:4}
この可視化システムによって,現実の動きとの比較が容易になり特徴量の抽出を行いやすくなった.

調理工程のラベルは撮影した動画から推測し手入力している.
動画との同期は3D空間での可視化システムで見比べて行っている.
特に変化が見られたY座標軸の移動距離と推定した切ったタイミングを表したグラフが次の図\ref{fig:5}である.

\figimage{images/fig5.pdf}{70}{推定結果のグラフ}{fig:5}

まずは切った回数の推定を行う.
動画によるラベリングとデータより,切る際には一度奥(+)方向へ移動してから手前(-)方向へ移動しているとわかる.
そのため今回は極値検出を用いて切った回数の推定を行う.
閾値を2cmと定め,閾値以上となった極大値から極小値までで1回切ったと推定する.
その結果を図\ref{fig:5}で「切り始め」と「切り終わり」として表している.
結果から切ったと推定した部分とラベリングを行なった区間が概ね一致した.
以上より切る動作を推定することができたと考えられる.
しかし包丁を持った所と置いたところも切っている部分と判定されてしまう場合があるため,あらかじめ包丁を持ってもらってからセンシングを始めるなど改善させる必要がある.

次に平均ペースを求める.
切るペースは1回1回の間の秒数を利用する.
また,その際の平均角度も特徴料として使用する.
今回のデータの特徴量を表にしたものを次の表\ref{table:1}に示す.
\begin{table}[ht]
    \centering
    \caption{特徴量}
    \label{table:1}
    \resizebox{70mm}{!}{ 
        \begin{tabular}{ccc}
            \hline\hline
            切った回数 & かかった時間 & 角度の平均 \\
            \hline
            1 & C1 & R1 \\
            2 & C2 & R2 \\
            3 & C1 & R1 \\
            4 & C2 & R2 \\
            5 & C1 & R1 \\
            \hline
            全体の平均 & C1 & R1 \\
            \hline
        \end{tabular}
    }
\end{table}
表\ref{table:1}の結果から得られた特徴量から次のような評価の基準になると考えられる.
まず切った回数や切っている最中の角度から刺身の厚さ,平均ペースから手際の良さなどを算出できると考えられる.
また,今回は参照していないが距離に変換する前の線形加速度を参照し一回の切り込み時の包丁のブレ等も算出することができると考えられる.

