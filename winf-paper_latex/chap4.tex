\section{刺身の平造り動作の分析と比較}
分析と比較について切る工程を例にして,実際にどのような特徴が見られるかを調べた.
\subsection{調査内容}
センシング方法としてスマートウォッチ(PixelWatch2)
を包丁を握る側の腕に装着し自作したアプリで加速度と角速度を収集した.
被験者には魚の切り身を刺身にしてもらった.
刺身を切る動作を分析の対象にした理由は,刃を入れる角度による身の厚みの変化や
包丁を引く回数で断面の滑らかさなどの評価基準が考えられたからである.
また,データのラベリングのために動画を撮影しながらデータ収集を行った.
\subsection{データの分析}
結果として得られた包丁の動作の一例を表したグラフを次の図に示す.

\figimage{images/fig1.pdf}{60}{グラフ(代打)}{fig:3}

このデータは3章で述べた推定手法を用いて求めた端末の回転量と移動を表している.

3次元的な動きを扱うにあたり扱うグラフで可視化してもわかりずらい.
そこで図\ref{fig:4}のような求めた回転と移動を3D空間での可視化を行うシステムを作成した.

\figimage{images/fig4.pdf}{60}{3D空間での可視化システム}{fig:4}

この可視化システムによって,現実の動きとの比較が容易になり特徴量の抽出を行いやすくなった.

調理工程のラベルは撮影した動画から推測し手入力している.
動画との同期は3D空間での可視化システムで見比べることで行っている.

ラベリングされている部分が実際に刺身を切り分けている部分である.
特に変化が見られたY座標軸の変化量のみを抜き出したグラフが次の図である.

\figimage{images/fig1.pdf}{60}{Y軸グラフ(代打)}{fig:5}

まずは切った回数の推定を行う.
動画によるラベリングとデータより,切る際には一度奥(+)方向へ移動してから手前(-)方向へ移動しているとわかる.
そのため今回は極値検出を用いて切った回数の推定を行う.
閾値をxxxと定め,閾値以上となった極大値から極小値までで1回切ったと推定する.
また,極大値から極小値までの時刻データの差分から一回の切り込みにかかった時間,
極大値から極小値までの時刻データの区間内の角度データから切り込んだ際の角度の推定を行う.

\subsection{データの比較}