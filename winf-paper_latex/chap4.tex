\section{評価実験}
本実験では3章で述べた特徴量抽出手法を刺身を切り分ける動作に対して行い,抽出した特徴量が調理技能の評価に利用できるかの確認を目的とする.
実験内容として刺身を切り分ける動作のセンシングを行う.
取得したデータをもとに3次元的な動きを推定し,特徴量の抽出を行う.
抽出した特徴量から調理技能の評価を行う.
本章では各節において実験の流れについて述べる.
\subsection{実験設定}
加速度と角速度を収集する自作アプリを起動したスマートウォッチ(PixelWatch2)を包丁を握る側の腕に装着しセンシングを行う.
被験者は10cmほどの魚の柵を刺身に切り分ける調理を行う.
刺身を切る動作は刃を入れる角度による断面の変化や包丁を引く回数から切り分けた身の厚さなどの複数の評価基準が得られる.

また,実際に切った区間のラベリングのために動画を撮影しながらデータ収集を行った.
後ほど切った区間のラベリングと推定した切っている区間が一致するかの確認に使用する.
\subsection{実験結果}
結果として得られた包丁の動作の一例を表したグラフを次の図\ref{fig:4}に示す.
\figimage{images/fig4.pdf}{80}{刺身を切る際の3次元的な動作のグラフ}{fig:4}
このデータは3章で述べた推定手法を用いて求めた端末の回転量から変換した角度と移動を表している.
誤差の軽減のためにハイパスフィルタを使用した.
赤くラベリングされている部分が刺身を切り分けている動作である.
また,青くラベリングされている部分は包丁を持つ動作と置く動作である.

3次元的な動きを扱うにあたりグラフで可視化しても分かりづらいため,手首の動きを画面上に再現するシステムを作成した.
作成した回転と移動を3次元空間での可視化を行うシステムが図\ref{fig:5}である.
\figimage{images/fig5.pdf}{80}{3次元空間での可視化システム}{fig:5}
この可視化システムによって,現実の動きとの比較が容易になり抽出しやすくなる.

切り分けている動作と準備動作のラベルは撮影した動画から手入力している.
動画との同期は3次元空間での可視化システムで見比べて行う.
特に変化が見られたY座標軸の移動距離と推定した切ったタイミングを表したグラフが図\ref{fig:6}である.

\figimage{images/fig6.pdf}{80}{推定結果のグラフ}{fig:6}

まずは切った回数の推定を行う.
動画によるラベリングとデータにより,切る際には一度奥(+)方向へ移動してから手前(-)方向へ移動しているとわかる.
そのため今回は極値検出を用いて切った回数の推定を行う.
切る際にメジャーを用い手首の移動距離を調べ,一番動かしていなかった切り込みを基準に閾値を2cmと定めた.
閾値以上となった極大値から次の極大値までの区間内で一番値が小さな極小値までで1回切ったと推定する.
その結果を図\ref{fig:6}で「切り始め」と「切り終わり」として表している.
結果から切ったと推定した部分とラベリングを行なった区間が一致した.
以上より切る動作を推定できた.
しかし包丁を持つ動作と置く動作も切っている部分と判定されてしまう場合があるため,あらかじめ包丁を持ってからセンシングを始めるなど改善する必要がある.
次に平均ペースを求める.
切るペースは一回毎の間の秒数を利用する.
また,その際のRoll(x)軸での平均角度も特徴量として使用する.

包丁を持つ動作と置く動作を除いたデータの特徴量を次の表\ref{table:1}に示す.
\begin{table}[ht]
    \centering
    \caption{特徴量}
    \label{table:1}
    \resizebox{70mm}{!}{ 
        \begin{tabular}{ccc}
            \hline\hline
            切った回数 & かかった時間(s) & 角度の平均(deg) \\
            \hline
            1 & 1.61 & -27.42 \\
            2 & 3.30 & -24.51 \\
            3 & 3.38 & -23.70 \\
            4 & 3.62 & -22.95 \\
            5 & 3.70 & -21.84 \\
            \hline
            全体の平均 & 3.12 & -24.08 \\
            \hline
        \end{tabular}
    }
\end{table}
表\ref{table:1}の結果から得られた特徴量は次のような評価の基準になる.
まず切った回数から刺身の厚さが算出できる.
今回のデータの場合5回切っているため6枚の刺身ができている.
切った柵が10cmなため10cmを6等分した結果一枚約1.6cmとわかる.
次に切るペースから手際の良さが算出できる.
今回の場合は平均ペースが3.12sとわかる.
最後に切る際の角度から切り方が推定できる.
今回の場合は柵に対して垂直な状態から右に24°ほど角度をつけて切っている.
そのため厚めに切る手法である平造りであると推定できる.
実験より本研究で抽出した特徴量は刺身を切る動作において調理技能の評価に利用できると考える.

